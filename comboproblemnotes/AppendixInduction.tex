\chapter{Mathematical Induction}\label{Induction}
\section{The Principle of Mathematical Induction}
\subsection{The ideas behind mathematical induction}
There is a variant of one of  the bijections we used to prove the Pascal
Equation that comes up in counting the subsets of a set. In the next
problem it will help us compute the total number of subsets of a set,
regardless of their size.  Our main goal in this problem, however, is to
introduce some ideas that will  lead us to one of the most powerful proof
techniques in combinatorics (and many other branches of mathematics), the
principle of mathematical induction.
\bp
\item \label{subsetsbysmallestcounterexample}
\begin{enumerate}
\item Write down a list of the subsets of $\{ 1, 2 \}$.
Don't forget the empty set! Group the sets containing containing 2
separately from the others.
\solution{$\emptyset$, $\{1\}$,\qquad $\{2\}$, $\{1, 2\}$.}
\item Write down a list of the subsets of $\{ 1, 2, 3 \}$. Group the sets
 containing 3 separately from the others.
\solution{$\emptyset$, $\{1\}$, $\{2\}$, $\{1, 2\}$,\qquad $\{3\}$,
$\{1,3\}$, $\{2,3\}$, $\{1,2,3\}$.}

\item Look for a natural way to match up the subsets containing 2 in
Part (a) with those not containing 2. Look for a way to match up
the subsets containing 3 in Part (b) containing 3 with those not
containing 3.
\solution{Adjoin $2$ to each subset not containing $2$ and you get each set
containing 2.  Adjoin 3 to each subset not containing 3, and you get each
subset containing 3.}

\item On the basis of the previous part, you should be able to find a
bijection between the collection of subsets of $\{1, 2, \ldots , n \}$
containing $n$ and those not containing $n$. (If you are having difficulty
figuring out the bijection, try rethinking Parts (a) and (b), perhaps by
doing a similar exercise with the set $\{1,2,3,4\}$.) Describe the
bijection (unless you are very familiar with the notation of sets, it is
probably easier to describe to describe the function in words rather than
symbols) and explain why it is a bijection. Explain why the number of
subsets of
$\{ 1, 2, \ldots , n \}$ containing
$n$ equals the number of subsets of $\{ 1, 2, \ldots, n-1 \}$. 
\solution{If we adjoin $n$ to the subsets not containing $n$ we get the
subsets containing $n$.  This is a bijection because if we start with two
different sets, adjoining $n$ to them can't make them the same, and every
subset $S$ containing $n$ must arise in this way from the set $S-\{n\}$ not
containing $n$.}

\item Parts (a) and (b) suggest
strongly that the number of subsets of a $n$-element set is $2^n$. In
particular, the empty set has $2^0$ subsets, a one-element set has $2^1$
subsets, itself and the empty set, and in Parts a and b we saw that
two-element and three-element sets have $2^2$ and $2^3$ subsets
respectively. So there are certainly some values of $n$ for which an
$n$-element set has $2^n$ subsets. One way to prove that an $n$-element
set has $2^n$ subsets for all values of $n$ is to argue by contradiction.
For this purpose, suppose there is a nonnegative integer $n$ such that an
$n$-element set doesn't have exactly $2^n$ subsets. In that case there may
be more than one such $n$. Choose $k$ to be the smallest such $n$. Notice
that $ k -1$ is still a positive integer, because $k$ can't be 0, 1, 2,
or 3. Since $k$ was the smallest value of $n$ we could choose to make the
statement
``An $n$-element set has $2^n$ subsets" false, what do you know about the
number of subsets of a $(k - 1)$-element set? What do you know about the
number of subsets of the $k$-element set $\{ 1, 2, \ldots, k \}$ that
don't contain $k$? What do you know about the number of subsets of $\{ 1,
2, \ldots,  k \}$ that do contain $k$? What does the sum principle tell
you about the number of subsets of $\{ 1, 2, \ldots, k \}$? Notice that
this contradicts the way in which we chose $k$, and the only assumption
that went into our choice of $k$ was that ``there is a nonnegative integer
$n$ such that an $n$-element set doesn't have exactly $2^n$ subsets."
Since this assumption has led us to a contradiction, it must be false.
What can you now conclude about the statement ``for every nonnegative
integer $n$, an n-element set has exactly $2^n$ subsets?"
\solution{We know that the number of subsets of a $(k-1)$-element set is
$2^{k-1}$.  The number of subsets of $\{1,2,\ldots,k\}$ that do not contain
$k$ is the number of subsets of the $k-1$-element set $\{1,2,\ldots,
k-1\}$, so we know this number is $2^{k-1}$.  We know that  the number of
subsets that do contain $k$ equals the number that don't, so the number
that do contain $k$ is also $2^{k-1}$.  The sum principle tells us that the
number os subsets of $\{1,2,\ldots, k\}$ is $2^{k-1}+2^{k-1}=2^k$.  We can
conclude that the statement ``for every nonnegative integer $n$, an
$n$-element set has exactly $2^n$ subsets" is true.}
\end{enumerate}

\item The expression\label{sumodd}
$$1+3+5+\cdots+2n-1$$
is the sum of the first $n$ odd integers.  Experiment a bit with the sum
for the first few positive integers and guess its value in terms of $n$. 
Now apply the technique of Problem
\ref{subsetsbysmallestcounterexample} to prove that you are right.
\solution{We guess that $1+3+5+\cdots+2n-1=n^2$.  Clearly this is true when
$n$ is 1, 2, or 3.  Suppose there is an $n$ for which this formula is not
true, and let $k$ be the smallest such $n$.  Then $1+3+5+\cdots+2(k-1)-1 =
(k-1)^2$.  Simplifying, $1+3+5+\cdots+2k-3 =
(k-1)^2$.  Now suppose we add $2k-1$ to both sides of this equation.  Then
we get $$1+3+5+\cdots+2k-3 +2k-1 =
(k-1)^2+2k-1 = k^2-2k+1+2k-1=k^2.$$
But this is a contradiction, because we assumed that $k$ was the
smallest value of $n$ for which the sum on the left is not $n^2$. 
Therefore the assumption that there is an $n$ for which $1+3+5+\cdots+2n-1
\not=n^2$ must be false, so the equation $1+3+5+\cdots+2n-1=n^2$ must be
true for all positive integers $n$.}

\ep

In Problems \ref{subsetsbysmallestcounterexample} and \ref{sumodd} our
proofs had several distinct elements.  We had a statement involving an
integer $n$.  We knew the statement was true for the first few
nonnegative integers in Problem \ref{subsetsbysmallestcounterexample} and
for the first few positive integers in problem \ref{sumodd}.  We wanted
to prove that the statement was true for all nonnegative integers in
Problem \ref{subsetsbysmallestcounterexample} and for all positive
integers in Problem \ref{sumodd}.  In both cases we used the method of
proof by contradiction; for that purpose we assumed that there was a value
of
$n$ for which our formula wasn't true.  We then chose $k$ to be the
smallest value of $n$ for which our formula wasn't true.  This meant that
when $n$ was $k-1$, our formula was true, (or else that $k-1 $ wasn't a
nonnegative integer in Problem \ref{subsetsbysmallestcounterexample} or
that $k-1$ wasn't a positive integer in Problem \ref{sumodd}).  What we
did next was the crux of the proof.  We showed that the truth of our
statement for
$n=k-1$ implied the truth of our statement for $n=k$.  This gave us a
contradiction to the assumption that there was an $n$ that made the
statement false.  In fact, we will see that we can bypass entirely the
use of proof by contradiction.  We used it to help you discover the
central ideas of the technique of proof by mathematical induction.

The central core of mathematical induction is the proof
that the truth of a statement about the integer $n$ for $n=k-1$ implies
the truth of the statement for $n=k$.  For example, once we know that a
set of size 0 has $2^0$ subsets, if we have proved our implication, we
can then conclude that a set of size 1 has $2^1$ subsets, from which
we can conclude that a set of size 2 has $2^2$ subsets, from which we can
conclude that a set of size 3 has $2^3$ subsets, and so on up to a set of
size $n$ having $2^n$ subsets for any nonnegative integer $n$ we choose. 
In other words, although it was the idea of proof by contradiction that
led us to think about such an implication, we can now do without the
contradiction at all.  What we need to prove a statement about $n$ by
this method is a place to start, that is a value $b$ of $n$ for which we
know the statement to be true, and then a proof that  the truth of our
statement for $n=k-1$ implies the truth of the statement for $n=k$
whenever $k>b$.

\subsection{Mathematical induction}  The {\bf principle of mathematical
induction}\index{mathematical induction!principle of}\index{principle of
mathematical induction}\index{induction!mathematical, the principle of}
states that
\begin{quote}
In order to prove a statement about an integer $n$, if we
can\begin{enumerate}
\item Prove the statement when $n=b$, for some fixed integer $b$
\item Show that the truth of the statement for $n=k-1$ implies the truth
of the statement for $n=k$ whenever $k>b$,
\end{enumerate}
then we can conclude the statement is true for all integers $n\ge
b$.\end{quote} As an example, let us return to Problem
\ref{subsetsbysmallestcounterexample}.  The statement we wish to prove is
the statement that ``A set of size $n$ has $2^n$ subsets."  

\begin{quote}Our statement
is true when $n=0$, because a set of size 0 is the empty set and the
empty set has $1=2^0$ subsets. (This step of our proof is called a {\em
base step}.)

Now suppose that $k>0$ and every set with $k-1$ elements has $2^{k-1}$
subsets.  Suppose $S=\{a_1,a_2,\ldots a_k\}$ is a set with $k$ elements. 
We partition the subsets of $S$ into two blocks.  Block $B_1$ consists of
the subsets that do not contain $a_n$ and block $B_2$ consists of the
subsets that do contain $a_n$.  Each set in $B_1$ is a subset of
$\{a_1,a_2,\ldots a_{k-1}\}$, and each subset of $\{a_1,a_2, \ldots
a_{k-1}\}$ is in $B_1$.  Thus $B_1$ is the set of all subsets of
$\{a_1,a_2,\ldots a_{k-1}\}$.  Therefore by our assumption in the first
sentence of this paragraph, the size of $B_1$ is $2^{k-1}$.  Consider the
function from $B_2$ to $B_1$ which takes a subset of $S$ including $a_n$
and removes $a_n$ from it.  This function is defined on $B_2$, because
every set in $B_2$ contains $a_n$.  This function is onto, because if
$T$ is a set in $B_1$, then $T\cup \{a_k\}$ is a set in $B_2$ which the
function sends to $T$.  This function is one-to-one because if $V$ and
$W$ are two different sets in $B_2$, then removing $a_k$ from them gives
two different sets in $B_1$.  Thus we have a bijection between $B_1$ and
$B_2$, so $B_1$ and $B_2$ have the same size.  Therefore by the sum
principle the size of
$B_1\cup B_2$ is $2^{k-1} +2^{k-1}=2^k$.  Therefore $S$ has $2^k$
subsets.  This shows that if a set of size $k-1$ has $2^{k-1}$ subsets,
then a set of size $k$ has $2^k$ subsets.  Therefore by the principle of
mathematical induction, a set of size $n$ has $2^n$ subsets for every
nonnegative integer $n$.
\end{quote}

The first sentence of the last paragraph is called the {\em inductive
hypothesis}.  In an inductive proof we always make an inductive
hypothesis as part of proving that the truth of our statement when
$n=k-1$ implies the truth of our statement when $n=k$.  The last
paragraph itself is called the {\em inductive step} of our proof.  In an
inductive step we derive the statement for $n=k$ from the statement for
$n=k-1$, thus proving that the truth of our statement when $n=k-1$
implies the truth of our statement when $n=k$.  The last sentence in the
last paragraph is called the {\em inductive conclusion}.  All inductive
proofs should have a base step, an inductive hypothesis, an inductive
step, and an inductive conclusion.  

There are a couple details worth noticing.  First, in this problem, our
base step was the case $n=0$, or in other words, we had $b=0$.  However,
in other proofs,
$b$ could be any integer, positive, negative, or 0.  Second, our proof
that the truth of our statement for $n=k-1$ implies the truth of our
statement for
$n=k$ required that $k$ be at least 1, so that there would be an element
$a_k$ we could take away in order to describe our bijection.  However,
condition (2) of the principle of mathematical induction only requires
that we be able to prove the implication for $k>0$, so we were allowed to
assume $k>0$.
\bp \item Use mathematical induction to prove your formula from
Problem \ref{sumodd}.
\ep
\subsection{Proving algebraic statements by induction}
\bp
\item Use mathematical induction to prove the well-known formula that for
all positive integers $n$,
$$1+2 + \cdots +n = {n(n+1)\over 2}.$$
\solution{When $n=0$, $0=0(0+1)/2$, so our formula holds.  Now suppose
that$k>0$ and that our formula holds when $n=k-1$, so that
$1+2+\cdots+k-1=(k-1)k/2$.  Add $k$ to both sides of this equation to get
\begin{eqnarray*}
1+2+\cdots+(k-1)+k&=& (k-1)k/2 +k\\ &=& k^2/2-k/2+k\\&=&
k^2/2+k/2\\&=&k(k+1)/2.
\end{eqnarray*}
 Thus the truth of our formula for $n=k-1$ implies its
truth for $n=k$.  Therefore by the principle of mathematical induction, our
formula holds for all nonnegative integers $n$.}

\item Experiment with various values of $n$ in the sum
$${1\over 1\cdot2}+{1\over 2\cdot3} + {1\over3\cdot
4}+\cdots+{1\over n\cdot (n+1)} = \sum_{i=1}^n {1\over i\cdot(i+1)}.$$ 
Guess a formula for this sum and prove your guess is correct by induction.
\solution{We guess the formula 
$$\sum_{i=1}^n{1\over i(i+1)} = {n\over
n+1}.$$  When $n=1$ this formula says that ${1\over 1\cdot2}={1\over 1\cdot
2}$, so our formula holds when $n=1$.  Now assume that $k>1$ and that our
formula holds when $n=k-1$.  Then
$$\sum_{i=1}^{k-1} {1\over i(i+1)}= {k-1\over k}.$$
Adding $1\over k(k+1)$ to both sides of this equation gives us
\begin{eqnarray*}
\sum_{i=1}^{k-1} {1\over i(i+1)}+{1\over k(k+1)} &=& {k-1\over k}+{1\over
k(k+1)}\\
\sum_{i=1}^k {1\over i(i+1)}&=&{(k-1)(k+1)\over k(k+1)}+{1\over k(k+1)}\\
&=&{k^2 -1 +1\over k(k+1)} \\&=& {k\over k+1}.
\end{eqnarray*}  
Thus whenever our formula is true with $n=k-1$, it is true with $n=k$ as
well.  Therefore by the principle of mathematical induction, our formula is
true for all positive integers.}

\item For large values of $n$, which is larger, $n^2$ or $2^n$?  Use
mathematical induction to prove that you are correct.
\solution{We note that $0^2=0$, while $2^0=1$, that $1^2=1$, while
$2^1=2$, that $2^2=4$, while $2^2=4$, that $3^2=9$ while $2^3=8$, that
$4^2=16$ while $2^4=16$, and $5^2=25$ while $2^5=32$.  We suspect that
$2^n>n^2$ for $n\ge 5$, so we try to prove this by induction.  We have
already shown that $2^5>5^2$.  Now suppose that $k>5$ and
$2^{k-1}>(k-1)^2$.  Then $2^k=2\cdot2^{k-1}>2(k-1)^2=2k^2-4k +1$. Now since
$k>5$, $k^2>5k$, so that $k^2-4k+1=k^2+k^2-4k+1>k^2+5k-4k+1=k^2+k+1>k^2$. 
Thus for $k>5$, the statement $2^{k-1}>(k-1)^2$ implies the statement
$2^k>k^2$.  Therefore, by the principle of mathematical induction,
$2^n>n^2$ for all $n\ge 5$.}

\item What is wrong with the following attempt at an inductive proof that
all  integers in any consecutive set of $n$ integers are equal for
every positive integer $n$?  For an arbitrary integer $i$, all integers
from $i$ to $i$ are equal, so our statement is true when $n=1$.  Now
suppose $k>1$ and all integers in any consecutive set of $k-1$ integers
are equal.  Let $S$ be a set of $k$ consecutive integers.  By the
inductive hypothesis, the first $k-1$ elements of $S$ are equal and the
last $k-1$ elements of $S$ are equal.  Therefore
all the elements in the set $S$ are equal.  Thus by the principle of
mathematical induction, for every positive $n$, every $n$ consecutive
integers are equal.
\solution{One possible value of $k$ that is greater than 1 is 2.  When we
have a set $S$ of two elements and we argue that the first $k-1$ elements
are equal and the last $k-1$ elements are equal, we cannot conclude from
those equalities that all elements of $S$ are equal, because there is no
overlap among the first $k-1=1$ elements of $S$ and the last $k-1=1$
elements of $S$.  Thus our inductive step does not cover the possibility
that $k=2$.  Therefore our inductive step does not show that the truth of
our statement for $n=k-1$ implies the truth of our statement for $n=k$ for
{\bf all} integers $n>1$.  Therefore the principle of mathematical
induction does not apply.}
\ep
\subsection{Strong Induction}
One way of looking at the principle of mathematical induction is that it
tells us that if we know the ``first" case of a theorem and we can derive
each other case of the theorem from a smaller case, then the theorem is
true in all cases.  However the particular way in which we stated the
theorem is rather restrictive in that it requires us to derive each
case from the immediately preceding case.  This restriction is
not necessary, and removing it leads us to a more general statement of
the principal of mathematical induction which people often call the {\bf
strong principle of mathematical induction}.  It states:
\begin{quote}In order to prove a
statement about an integer $n$ if we can
\begin{enumerate}
\item prove our statement when $n=b$ and
\item prove that the statements we get with $n=b$, $n=b+1$, \ldots
$n=k-1$ imply the statement with $n=k$,
\end{enumerate}
then our statement is true for all integers $n\ge b$.
\end{quote} 
\bp
\item What postage do you think we can make with five and six cent
stamps?  Is there a number $N$ such that if $n\ge N$, then we can make
$n$ cents worth of postage?
\solution{We can make 10, 11, and 12 cents in postage, but not 13 cents. 
We can also make 15, 16, 17, and 18, but not 19 cents.  However when we try
starting with 20 cents, we can make 20, 21, 22, 23, 24, 25, 26,
27,...cents, and so it seems for all $n\ge 20$, we can make $n$ cents in
stamps.  Once we know we can make 20 cents through 24 cents, by adding 5
cents to each of these we can get 25 through 29 cents, and so we expect to
be able to keep going.  However making 29 cents does not depend on our
ability to make 28 cents; rather we know we can make 29 cents because we
know we can make 24 cents and $24+5=29$ or we know we can make $23$ cents
and $23+6=29$.  Thus it certainly seems as if for all $n\ge 20$ we can make
$n$ cents in postage.}
\ep

You probably see that we can make $n$ cents worth of postage as long as
$n$ is at least 20.  However you didn't try to make 26 cents in postage
by working with 25 cents; rather you saw that you could get 20 cents and
then add six cents to that to get 26 cents.  Thus if we want to prove by
induction that we are right that if $n\ge 20$, then we can make $n$ cents
worth of postage, we are going to have to use the strong version of the
principle of mathematical induction.  

We know that  we can make 20 cents
with four five-cent stamps.  Now we let $k$ be a number greater than 20,
and assume that it is possible to make any amount between 20 and $k-1$
cents in postage with five and six cent stamps.  Now if $k$ is less than
25, it is 21, 22, 23, or 24.  We can make 21 with three fives and one
six.  We can make 22 with two fives and two sixes, 23 with one five and
three sixes, and 24 with four sixes.  Otherwise $k-5$ is  between 20 and
$k-1$ (inclusive) and so by our inductive hypothesis, we know that $k-5$
cents can be made with five and six cent stamps, so with one more five
cent stamp, so can $k$ cents.  Thus by the (strong) principle of
mathematical induction, we can make $n$ cents in stamps with five and six
cent stamps for each $n\ge 20$.

Some people might say that we really
had five base cases,
$n=20$, 21, 22, 23, and 24, in the proof above and once we had proved
those five consecutive base cases, then we could reduce any other case to
one of these base cases by successively subtracting 5.  That is an
appropriate way to look at the proof.  A logician would say that it
is also the case that, for example, by proving we could make 22 cents, we
also proved that if we can make 20 cents and 21 cents in stamps, then we
could also make 22 cents.  We just didn't bother to use the assumption
that we could make 20 cents and 21 cents!  So long as one point of view
or the other satisfies you, you are ready to use this kind of argument in
proofs.


\bp
\item A number greater than one is called prime if it has no factors other
than itself and one. Show that each positive number is either a prime or
a power of a prime or a product of powers of prime numbers.
\solution{We note that $1=2^0$, so 1 is a power of a prime.  Now suppose
that all positive numbers less than $n$ are primes, powers of primes, or
products of powers of primes. If $n$ has no proper factors, it is a prime. 
If it does have proper factors, say $n=mk$, both factors are less than $n$
and greater than 1.  Therefore each factor is a prime, a power of a prime,
or a product of powers of primes.  When we multiply $m$ and $k$ together,
the result will still be a power of a prime or a product of powers of
primes. Thus the  statement that all positive numbers less than
$n$ are primes, powers of primes, or products of powers of primes implies
 the statement that $n$ is a prime, a power of a prime, or a product of 
powers of primes.  Therefore by the strong principle of mathematical
induction, all positive numbers are either primes, powers of primes, or
products of powers of primes.}

\item Show that the number of prime factors of a positive number $n\ge 2$
is less than or equal to $\log_2 n$.  (If a prime occurs to the $k$th
power in a factorization of $n$, you can consider that power as $k$ prime
factors.)  (There is a way to do this by induction and a way to do it
without induction.  It would be ideal to find both ways.)
\solution{First, we will prove this by induction.  The number of prime
factors of $2$ is 1, which is less than or equal to $\log_2 2=1$. Now
assume that the number of prime factors of any number $k$ greater than 1 and
less than $n$ is no more than $\log_2 k$.  If $n$ is prime, then its number
of prime factors is less than or equal to $\log_2 n$.  Otherwise $n$ is a
product of two factors, $n=mk$.  Then by our inductive hypothesis,  the
number of prime factors of $m$ is less than or equal to $\log_2 m$ and the
number of prime factors of $k$ is less than or equal to $\log_2 k$.  But
the number of prime factors of the product is the sum of the number of
prime factors of each factor, so the number of prime factors of $n$ is no
more than $\log_2 m +\log_2 k=\log_2 mk= \log_2 n$.  Thus statement that the
number of prime factors of any number $k$ between 2 and $n-1$ inclusive is
no more than $\log_2 k$, implies the statement that the number of prime
factors of $n$ is no more than $\log_2 n$.  Therefore, by the principle of
mathematical induction, the number of prime factors of $n$ is less than or
equal to
$\log_2 n$ for all integers $n\ge 2$.

For a noninductive proof, note that all factors of $n$ are at least 2.  If
$n$ is a power of two, then the number of times $2$ is a factor of $n$ is
exactly
$\log_2 n$. But if $n$ is not a power of 2, we still have that
$2^{\log_2 n}=n$, so the product of $\lceil\log_2 n\rceil$ numbers
including some greater than 2 must be greater than $n$. Therefore, the
number of prime factors of $n$ is no more than
$\log_2 n$. Thus for all $n\ge2$,the number of prime factors of
$n$ must be less than or equal to
$\log_2 n$.}
\ep