\chapter{Exponential Generating Functions}\label{expogenfun}
\section{Indicator Functions}
When we introduced the idea of a generating function, we said that the
formal sum
  $$\sum_{i=0}^n a_ix^i$$
may be thought of as a convenient way to keep track of the sequence
$a_i$.  We then did quite a few examples that showed how combinatorial
properties of arrangements counted by the coefficients in a generating
function could be mirrored by algebraic properties of the generating
functions themselves.  The monomials $x^i$ are called {\em indicator
polynomials}\index{indicator polynomials}.  (They indicate the position of
the coefficient
$a_i$.)  One example of a generating function is given by 
$$(1+x)^n = \sum_{i=0}^\infty {n\choose i}x^i.$$
Thus we say that $(1+x)^n$  is the generating function for the
binomial coefficients $n \choose i$.  The notation tells us that we are
assuming that only $i$ varies in the sum on the right, but that the
equation holds for each fixed integer $n$.  This is implicit when we say
that $(1+x)^n$ is the generating function for $n\choose i$, because we
haven't written $i$ anywhere in $(1+x)^n$, so it is free to vary.

Another example of a generating function is given by
$$x^{\underline{n}} = \sum_{i=0}^\infty s(n,i)x^i.$$
Thus we say that $x^{\underline{n}}$ is the generating function for the
Stirling numbers of the first kind, $s(n,i)$.  There is a similar
equation for Stirling numbers of the second kind, namely
$$x^n = \sum_{i=0}^\infty S(n,i)x^{\underline{i}}.$$
However with our previous definition of generating functions, this
equation would not give a generating function for the Stirling numbers
of the second kind, because $S(n,i)$ is not the coefficient of $x^i$.  If
we were willing to consider the falling factorial powers
$x^{\underline{i}}$ as indicator polynomials, then we could say that
$x^n$ is the generating function for the numbers $S(n,i)$ relative to
these indicator polynomials.  This suggests that perhaps different sorts
of indicator polynomials go naturally with different sequences of
numbers.  

The binomial theorem gives us yet another example. 
\bp
\itemm Write $(1+x)^n$ as a sum of multiples of
$x^i\over i!$ rather than as a sum of multiples of
$x^i$.\solution{$$(1+x)^n =
\sum_{i=0}^\infty {n\choose i}x^i=\sum_{i=0}^\infty {n!\over i!(n-i)!}x^i
=
\sum_{i=0}^\infty {n!\over (n-i)!}{x^i\over i!} = \sum_{i=0}^\infty
n^{\underline{i}} {x^i\over i!}.$$}
\ep

This example suggests that we could say that $(1+x)^n$
is the generating function for the falling factorial powers
$n^{\underline{i}}$ relative to the indicator polynomials $x^i\over i!$. 
In general, a sequence  of polynomials is called a family of {\bf
indicator polynomials} if there is one polynomial of each nonnegative
integer degree in the sequence.  Those familiar with linear algebra will
recognize that this says that a family of indicator polynomials form a
basis for the vector space of polynomials.  This means that each
polynomial way can be expressed as a sum of numerical multiples of
indicator polynomials in one and only one way.  One could use the
language of linear algebra to define indicator polynomials in an even
more general way, but a definition in such generality would not be useful
to us at this point.  

\section{Exponential Generating Functions}
We say that the expression $\sum_{i=0}^\infty a_i{x^i\over i!}$ is the
{\bf exponential generating function}\index{exponential generating
function}\index{generating function!exponential} for the sequence
$a_i$.  It is standard to use {\bf EGF}\index{EGF} as a shorthand for
exponential generating function.  In this context we call the generating
function
$\sum_{i=0}^n a_ix^i$ that we originally studied the {\bf ordinary
generating function}\index{generating function!ordinary}\index{ordinary
generating function} for the sequence
$a_i$.  You can see why we use the term exponential generating function
by thinking about the exponential generating function (EGF) for the all
ones sequence,
$$\sum_{i=0}^\infty 1{x^i\over i!} = \sum_{i=0}^\infty {x^i\over i!}
=e^x,$$
which we also denote by $\exp (x)$.  Recall from calculus that the usual
definition of
$e^x$ or $\exp(x)$ involves limits at least implicitly.  We work our way
around that by defining $e^x$ to be the power series $ \sum_{i=0}^\infty
{x^i\over i!}$.

\bp 

\itemm Find the EGF (exponential generating function) for the sequence
$a_n=2^n$.  What does this say about the EGF
for the number of subsets of an $n$-element set?
\solution{$\sum_{i=0}^\infty {2^i}{x^i\over i!}=e^{2x}$.  It says
that the EGF for subsets of an $n$-element set is $e^{2x}$.}

\itemm Find the EGF (exponential generating function) for the number of
ways to paint the $n$ streetlight poles that run along the north side of
Main Street in Anytown, USA using four colors.\label{paintinglightpoles}
\solution{The number of ways to paint the streetlights is $4^n$,
so the EGF is $\sum_{n=0}^\infty 4^n{x^n\over n!}=e^{4x}$.}

\item For what sequence is ${e^x-e^{-x}\over 2} =\cosh x$ the EGF
(exponential generating function)?
\solution{For the sequence $1-(-1)^n\over 2$ which, starting with $n=0$ is
the alternating sequence 0,1,0,1\ldots of zeros and ones.}

\itemes For what sequence is $\ln({1\over 1-x})$ the EGF?
\label{ln1over1-x}($\ln (y)$ stands for the natural logarithm of $y$. 
People often write
$\log(y)$ instead.) Hint:  Think of the definition of the logarithm as an
integral, and don't worry at this stage whether or not the usual laws of
calculus apply, just use them as if they do!  We will then define
$\ln({ 1-x})$ to be the power series you
get.\label{naturallog}\footnote{It is possible to define the derivatives
and integrals of power series by the formulas
$${d\over dx}
\sum_{i=0}^\infty b_ix^i = \sum_{i=1}^\infty ib_ix^{i-1}$$ and $$\int_0^x
\sum_{i=0}^\infty b_ix^i = \sum_{i=0}^\infty {b_i\over i+1}x^{i+1}$$
rather than by using the limit definitions from calculus.  It is then
possible to prove that the sum rule, product rule, etc. apply.  (There is
a little technicality involving the meaning of composition for power
series that turns into a technicality involving the chain rule, but it
needn't concern us at this time.)}
\solution{
\begin{eqnarray*}\ln({1\over 1-x}) =-\ln(1-x) &=&
\int_0^x {1\over 1-t}dt\\ &=&\int_0^x (1+t+t^2+\cdots)dt\\
&=&\sum_{i=1}^\infty {x^i\over i}\\ &=& \sum_{i=1}^\infty (i-1)!{x^i\over
i!}\end{eqnarray*} so the sequence is
$a_n = (n-1)!$.
}


\itemes What is the EGF for the number of
permutations of an $n$-element set?\label{exponentialpermutations}
\solution{$1\over 1-x$.}

\itemesi  What is the EGF for the number of ways
to arrange $n$ people around a round table?  Notice that we may think of
this as the EGF for the number of
permutations on $n$ elements that are cycles.\label{exponentialroundtable}
\solution{By Problem \ref{ln1over1-x} the EGF is $\ln{1\over 1-x}$.}

\itemesi What is the EGF $\sum_{n=0}^\infty
p_{2n}{x^{2n}\over (2n)!}$ for the number of ways $p_{2n}$ to pair up $2n$
people to play a total of
$n$ tennis matches (as in Problems
\ref{tennispairings1} and
\ref{tennispairings2})?\label{exponentialtennisparings}
\solution{Recall that $p_{2n} = (2n-1)(2n-3)\cdots 1= {(2n)!\over 2^n
n!}$.  Thus $$\sum_{n=0}^\infty
p_{2n}{x^{2n}\over (2n)!}= \sum_{n=0}^\infty {x^{2n}\over 2^n n!} =
\sum_{n=0 }^\infty {({x^2/2})^n\over n!} = e^{x^2/2} .$$}

\itemm What is the EGF for the sequence
$0,1,2,3,\ldots$?  You may think of this as the EFG for the number of ways
to select one element from an $n$ element set.  What is the EGF for the
number of ways to select two elements from an
$n$-element set?
\solution{\hspace*{-10pt}
$\sum_{n=0}^\infty{ nx^n\over n!}=\sum_{n=1}^\infty{
x^n\over(n-1)!}=x\sum_{n=1}^\infty{ x^{n-1}\over(n-1)!}=xe^x$. \\
$\sum_{n=0}^\infty{ n(n-1)x^n\over 2 n!}=\sum_{i=2}^\infty {x^n\over
2(n-2)!}=x^2e^x/2$.}

\itemes What is the EGF for the sequence
$1,1,\cdots,1,\cdots$?  Notice that we may think of this as the
EGF for the number of identity permutations
on an $n$-element set, which is the same as the number of permutations of
$n$ elements that are products of 1-cycles, or as the EGF for the number
of ways to select an
$n$-element set (or, if you prefer, an empty set) from an
$n$-element set.\label{allonessequence}
\solution{$e^x$.}

\itemm  What is the EGF for the number of ways
to select $n$ elements from a one-element set?  What is the EGF for the
number of ways to select a positive number $n$ of elements from a one
element set? 
\solution{$1+x$, $x$.}

\itemes What is the EGF for the number of partitions of a
$k$-element set into exactly one block?  (Hint:
is there a partition of the empty set into
exactly one block?)\label{oneblockpartitions}
\solution{$e^x-1$.}

\itemes What is the EGF for the number of ways
to arrange $k$ books on one shelf (assuming they all fit)?  What is the
EGF for the number of ways to arrange $k$
books on a fixed number
$n$ of shelves, assuming that all the books can fit on any one
shelf?\label{exponentialbookshelf}
\solution{$1\over 1-x$, $ \sum_{k=0}^\infty {n+k-1 \choose k}k!{x^k\over
k!} = (1-x)^{-n}$.}

\ep

  

\section{Applications to recurrences.} 
We saw that ordinary generating functions often play a role in solving
recurrence relations.  We found them most useful in the constant
coefficient case.  Exponential generating functions are useful in solving
recurrence relations where the coefficients involve simple functions of
$n$, because the $n!$ in the denominator can cancel out factors of $n$ in
the numerator.  

\bp 
\itemm  Consider the recurrence
$a_n=na_{n-1} +n(n-1)$.  Multiply both sides by $x^n\over n!$, and sum
from $n=2$ to $\infty$.  (Why do we sum from $n=2$ to infinity instead 
of $n=1$ or $n=0$?)  Letting
$y =
\sum_{i=0}^\infty a_ix^i$, show that the left-hand side of the equation
is $y-a_0 -a_1x$.  Express the right hand side in terms of $y$, $x$, and
$e^x$.  Solve the resulting equation for $y$ and use the result to get an
equation for
$a_n$.  (A finite summation is acceptable in your answer for $a_n$.)
\solution{\begin{eqnarray*}\sum_{n=2}^\infty a_n{x^n\over n!}
&=&\sum_{n=2}^\infty a_{n-1}{x^n\over (n-1)!} + \sum_{n=2}^\infty
{x^n\over (n-2)!}\\
&=& x\left(\sum_{n=0}^\infty a_n{x^n\over n!} - a_0\right) +
x^2\sum_{n=0}^\infty {x^n\over n!}\\
&=& x\left(\sum_{n=0}^\infty a_n{x^n\over n!}\right) - a_0x
+x^2e^x\end{eqnarray*} 
We sum from $n=2$ because otherwise we would have a factorial of a
negative number in the denominator. Thus $\sum_{n=0}^\infty a_n{x^n\over
n!} -a_0-a_1x =  x\left(\sum_{n=0}^\infty a_n{x^n\over n!}\right) - a_0x
+x^2e^x$, or
$$(1-x)\sum_{n=0}^\infty a_n{x^n\over n!}=a_0+a_1x-a_0x +x^2e^x.$$
This gives us
$$\sum_{n=0}^\infty a_n{x^n\over n!}={1\over 1-x}(a_0+a_1x-a_0x
+x^2e^x).$$
computing the coefficient of $a_n$ gives us $a_n = a_1 +\sum_{i=0}^{n-2}
{1\over i!}$. }

\itemesi The telephone company in a city has $n$ subscribers.  Assume a
telephone call involves exactly two subscribers (that is, there are no
calls to outside the network and no conference calls), and that the
configuration of the telephone network is determined by which pairs of
subscribers are talking.  Give a recurrence for the number
$c_n$ of configurations of the network.  )Hint: Person $n$ is either on
the phone or not.)  What are $c_0$ and $c_1$?  What are $c_2$ through
$c_6$?  Notice that we may think of a configuration of the telephone
network as a permutation that is a product of disjoint two-cycles (and
one-cycles\footnote{When we think of writing a
permutation as a product of disjoint cycles, we often do not include the
one-cycles in our notation because a one-cycle is an identity
permutation.  However any element not moved by the cycles we do write
down is in a one cycle, and those one cycles are implicit in the product
of cycles we do write down.}), that is, we may think of a configuration
as an involution in the symmetric group $S_n$.
\label{telephonenetwork}
 \solution{$c_n =(n-1)c_{n-2} + c_{n-1}$.  (The first term counts the
number of network configurations in which person $n$ is in a phone call
with someone else, and the second term counts the number of network
configurations in which person
$n$ is not in a phone call.) $c_0 =1$ and $c_1=1$, because there is only
one configuration of a network with 0 or one phones.)  Then $c_2 =2$,
$c_3 =2\cdot1 +2=4$, $c_4= 3\cdot2+4=10$, $c_5= 4\cdot4 +10 = 26$, and
$c_6 = 5\cdot 10+26=76$.
}

\itemesi Recall that a {\em derangement} of $[n]$ is a permutation of
$[n]$ that has no fixed points, or equivalently is a way to pass out
$n$ hats to their $n$ different owners so that nobody gets the correct
hat.  Use
$d_n$ to stand for the number of derangements of $[n]$.  We can think of
derangement of $[n]$ as a list of $1$ through $n$ so that
$i$ is not in the $i$th place for any $n$.  Thus in a derangement, some
number
$k$ different from $n$ is in position $n$.  Consider two cases: either
$n$ is in position $k$ or it is not.  Notice that in the second case, if
we erase position $n$ and replace $n$ by $k$, we get a derangement of
$[n-1]$.  Based on these two cases, find a recurrence for $d_n$.  What is
$d_1$?  What is $d_2$?  What is $d_0$?  What are $d_3$ through
$d_6$?\label{derangementrecurrence}
 \solution{
$d_n = (n-1)d_{n-1} + (n-1)d_{n-2}$.  $d_1 = 0$ and $d_2 = 1$.  Thus
$d_0$ must be 1 for our recurrence to be valid.  (For those familiar with
functions as sets of ordered pairs, the empty function is not only a
permutation, but it does not map $i$ to $i$ for any integer $i$, so it is
a derangement as well!  Thus the definition of a derangement also gives
us that $d_0=1$.)  $d_3= 2$, $d_4=3\cdot1+3\cdot 2=9$,
$d_5=4\cdot2+4\cdot 9 = 44$, and $d_6 = 5\cdot9+5\cdot44 = 256$.}

\subsection{Using calculus  with exponential generating functions}

\itemesi Your recurrence in Problem \ref{telephonenetwork} should be a
second order recurrence.  
\begin{enumerate}
\item Assuming that the left hand side is $c_n$ and
the right hand side involves $c_{n-1}$ and $c_{n-2}$, decide on an
appropriate power of $x$ divided by an appropriate factorial by which to
multiply both sides of the recurrence.  Using the fact that the
derivative of $x^n\over n!$ is $x^{n-1}\over (n-1)!$, write down a
differential equation for the EGF $T(x) =
\sum_{i=0}^\infty c_i{x^i\over i!}$.  Note that it makes sense to
substitute 0 for
$x$ in $T(x)$.  What is $T(0)$?  Solve your differential equation to find
an equation for
$T(x)$.\label{telephonenetworkEGF}
 \solution{
\begin{eqnarray*}\sum_{n=2}^\infty c_n{x^{n-1}\over (n-1)!}
\!&=&\!\sum_{n=2}^\infty(n-1) c_{n-2}{x^{n-1}\over (n-1)!} +
\sum_{n=2}^\infty c_{n-1}{x^{n-1}\over (n-1)!}\\ 
\sum_{n=1}^\infty c_n{x^{n-1}\over (n-1)!}- c_1 \!&=&\! x\sum_{n=2}^\infty
c_{n-2}{x^{n-2}\over (n-2)!} + \sum_{n=0}^\infty c_n{x^n\over n!} -c_0\\
T'(x)\!&=&\!xT(x) +T(x)
\end{eqnarray*}
$T(0) = c_0 =1$.  Then ${T'(x)\over T(x)} = x+1$, giving $\ln T(x)
=x^2/2+x+k$, and $T(x) =e^ke^{x+ x^2/2}=e^{x+x^2/2 }$, since $T(0)=1$. 
 }
\item Use your generating function to compute a formula for $c_n$.
 \solution
{
$T(x) = \sum_{i=0}^\infty (x+x^2/2)^i/i!$.  By the binomial
theorem, this gives $T(x) = \sum_{i=0}^\infty {\sum_{j=0}^i {i\choose
j}x^j\bigl({x^2\over2}\bigr)^{i-j}\over
i!}=\sum_{i=0}^\infty{\sum_{j=0}^i{i\choose j}x^{2i-j}2^{j-i}\over i!}$. 
Then the coefficient $c_n$ of $x^n$ is the sum over all $i$ and $j$ with
$2i-j=n$ and $j\le i$ of ${i\choose j}{n!\over i!}2^{j-i}$.  But if
$2i-j=n$, then $j= 2i-n$, and if $2i-n\le i$, then $i\le n$, so that $c_n
= {n!\over 2^n}\sum_{i=0}^n{i\choose 2i-n}{2^i\over i!}$.  Note that
$i\choose 2i-n$ is the same as $i\choose n-i$, which is 0 unless $i\ge
n/2$, which reduces our sum to $c_n = {n!\over
2^n}\sum_{i=\lceil n/2\rceil}^n{i\choose n-i}{2^i\over i!}$.
}
\end{enumerate}

\itemesi Your recurrence in  Problem \ref{derangementrecurrence}
should be a second order recurrence.\label{exponentialderangements}
\begin{enumerate}  
\item Assuming that the left-hand side is $d_n$ and the right hand side
involves $d_{n-1}$ and $d_{n-2}$, decide on an appropriate power of $x$
divided by an appropriate factorial by which to multiply both sides of
the recurrence.  Using the fact that the
derivative of $x^n\over n!$ is $x^{n-1}\over (n-1)!$, write down a
differential equation for the EGF $D(x) =
\sum_{i=0}^\infty d_i{x^i\over i!}$. What is $D(0)$?  Solve your
differential equation to find an equation for $D(x)$.
\item Use the equation you found for $D(x)$ to find an equation for
$d_n$.  Compare this result with the one you computed by inclusion and
exclusion. 
 \solution{
\begin{eqnarray*}
\sum_{n=2}^\infty d_n{x^{n-1}\over (n-1)!}
&=&\sum_{n=2}^\infty d_{n-1}{x^{n-1} \over (n-2)!} +\sum_{n=2}^\infty
d_{n-2}{x^{n-1}
\over (n-2)!} \\
\sum_{n=1}^\infty d_n{x^{n-1}\over (n-1)!} -d_1 &=&
x\sum_{n=2}^\infty d_{n-1}{x^{n-2}\over (n-2)!} +xD(x)\\
D'(x) -d_1 &=& xD'(x) +xD(x)\\
D'(x)(1-x) &=& xD(x)\\
D'(x)\over D(x) &=& x\over 1-x
\end{eqnarray*}
This gives us $\ln D(x) = -\ln(1-x) -x +c$, so that $D(x) = {1\over
1-x}e^{-x}e^c$.  Since $d_0=1$, we have $d(0)=1$, so $c=0$ and 
\begin{eqnarray*}D(x) &=& {e^{-x}\over 1-x}\\ &=&
\sum_{i=0}^\infty(-1)^i{x^i\over i!}\cdot\sum_{j=0}^\infty
x^j=\sum_{i=0}^\infty\left(\sum_{j=0}^i {(-1)^j\over j!}\right)
x^i.\end{eqnarray*} Thus $d_n = n!\sum_{j=0}^n{(-1)^j\over j!}$, as we
computed by inclusion and exclusion.
 }
\end{enumerate}

\ep

\section{The product principle}  One of our major tools for ordinary
generating functions was the product principle.  It is thus natural to
ask if there is a product principle for exponential generating functions.
In Problem \ref{exponentialbookshelf} you likely found that the
EGF for the number of ways of arranging $n$
books on one shelf was exactly the same as the EGF for the number of
permutations of $[n]$, namely $1\over 1-x$ or
$(1-x)^{-1}$.  Then using our formula from Problem \ref{bookcase} and the
generating function for multisets, you probably found that the EGF for
number of ways of arranging $n$ books on some fixed number
$m$ of bookshelves was $(1-x)^{-m}$.  Thus the EGF for $m$ shelves is a
product of $m$ copies of the EGF for one
shelf. 

\bp
\itemm In Problem \ref{paintinglightpoles} what would
the exponential generating function have been if we had asked for the
number of ways to paint the poles with just one color of paint? With two
colors of paint? What is the relationship between the EGF for painting the
$n$ poles with one color of paint and the EGF for painting the
$n$ poles with four colors of paint?  What is the
relationship between the EGF
for painting the $n$ poles with two colors of
paint and the EGF for painting
the poles with four colors of paint?\label{paintinglightpoles2}
\solution{With one color of paint, there would have been one way to paint
each pole so our EGF would be $\sum_{n=0}^\infty {x^n\over n!}$, or
$e^x$.  With two colors of paint, it would be $e^{2x}$ by analogy with
the solution to Problem \ref{paintinglightpoles}.  Thus the EGF for two
colors of paint would be the square of the EGF for one color of paint. 
The EGF for four colors of paint is the fourth power of the EGF for one
color of paint and the square of the EGF for two colors of paint.}
\ep

 In Problem \ref{telephonenetwork} you
likely found that the EGF for the number of
network configurations with $n$ customers was $e^{x+x^2/2}= e^x \cdot
e^{x^2/2}$.  In Problem \ref{allonessequence} you saw that the generating
function for the number of  permutations on $n$ elements that are
products of one cycles was
$e^x$, and in Problem \ref{exponentialtennisparings} you likely found that
the EGF for the number of tennis pairings of
$2n$ people, or equivalently, the number of permutations of $2n$ objects
that are products of $n$ two-cycles is $e^{x^2/2}$.  
\bp
\itemes What can you say about the relationship among the EGF for the
number of permutations that are products of disjoint two-cycles and
one-cycles, i.e., are involutions, the exponential generating function for
the number of permutations that are the product of disjoint two-cycles
only and the generating function for the number of permutations that are
the product of disjoint one cycles
only?\label{2cyclesand1cycles}\solution{The EGF for involutions in $S_n$
is the product of the EGF for the permutations in $S_n$ that are products
of $n/2$ two-cycles, and the EGF for permutations in $S_n$ that are
products of $n$ one-cycles.}
\ep

In Problem \ref{exponentialderangements} you likely found that the
EGF for the number of permutations of $[n]$
that are derangements is $e^{-x}\over 1-x$.  But every permutation is a
product of derangements and one cycles, because the permutation that
sends $i$ to $i$ is a one-cycle, so that when you factor a permutation as
a product of disjoint cycles, the cycles of size greater than one
multiply together to give a derangement, and the elements not moved by
the permutation are one-cycles.  
\bp
\itemes If we multiply the generating function for derangements times the
generating function for the number of permutations that are products of
one-cycles only, what EGF for what set of permutations do
we get? (Notice that there are two questions
here.)\label{derangementsand1cycles}\solution{We get the EGF $1\over 1-x$
for all permutations of $[n]$. Notice that any permutation is a product
of a derangement of the elements not fixed by the permutation times the
one cycles of elements that are fixed by the permutation.}
\ep

We now have four examples in which the EGF
for a sequence or a pair of objects is the product of the EGFs for the
individual objects making up the sequence or pair.

\bp 
\itemes What is the coefficient of $x^n\over n!$ in the product of two
EGFs $\sum_{i=0}^\infty a_i{x^i\over i!}$ and $\sum_{j=0}^\infty
b_j{x^j\over j!}$? (A summation sign is appropriate in your
answer.)\label{exponentialpp1}
 \solution{
$$\sum_{i,j\mbox{:\ }
i+j=n} n!{a_i\over i!}{b_j \over j!},$$ which can be better
written as $$\sum_{i=0}^n{n!\over i!(n-i)!} a_i
b_{n-i}=\sum_{i=0}^n{n\choose i} a_i b_{n-i}.$$  
 }
\ep

Our product principle for ordinary generating functions involved the idea
of a value function.  In particular, if we have a set $S$ of objects we
call a function $v$ from $S$ into the nonnegative
integers a {\em value function}.  Our combinatorial interpretation of the
product of ordinary generating functions in Problem
\ref{ProductPrincipleOGF} is the following theorem.

\begin{theorem} Suppose we have a set $S$ with a value function $v$ from
$S$ into the nonnegative integers and a set
$T$ with a value function
$u$
 from $T$ into the nonnegative integers.  If
$a_i$ is the number of objects $s$ in $S$ with value $v(s) =i$ and $b_j$
is the number of objects $t$ in $T$ with value $u(t)= j$, and $c_n$ is
the number of ordered pairs $(s,t)$ of objects in $S\times T$ with total
value $v(s) + u(t) =n$, then the ordinary  generating function for $c_n$
is the product of the ordinary generating function for $a_i$ and the
ordinary generating function for $b_j$. 
\end{theorem}


We ask if there is a similar
interpretation for the products of exponential generating functions we
have just seen.  In the case of painting streetlight poles in Problem
\ref{paintinglightpoles2}, what is important is not just how many
light poles are painted with each color, but which set of poles is painted
with each color.  Let us examine the relationship between the EGF for
painting poles with two colors, $e^{2x}$ and the EGF for painting poles
with four colors,
$e^{4x}$.  To be specific, the EGF for painting poles red and white is
$e^{2x}$ and the EGF for painting poles blue and yellow is $e^{2x}$.  To
decide how to paint poles with red, white, blue, and yellow, we can decide
which set of poles is to be painted with red and white, and which set of
poles is to be painted with blue and yellow.  Notice that the number of
ways to paint a set of poles with red and white depends only on the size
of that set, and the number of ways to paint a
set of poles with blue and yellow depends only on the size of that set. 
(It is a coincidence that the number of ways to paint a set of poles with
red and white equals the number of ways to paint the same set of poles
with blue and yellow.  The coincidence is the result of trying to keep our
example simple!)

\bp
\itemes Suppose that $a_i$ is the number of ways to paint a set of $i$
poles with red and white, and $b_j$ is the number of ways to paint a set
of
$j$ poles with blue and yellow.  In how many ways may we take a set $N$
of 
$n$ poles, divide it up into two sets $I$ and $J$ (using $i$ to stand for
the size of $I$ and $j$ to stand for the size of the set $J$, and
allowing $i$ and $j$ to vary) and paint the poles in $I$ red and white
and the poles in $J$ blue and yellow?  (Don't figure out formulas for
$a_i$ and $b_j$ to use in your answer; that will make it harder to get
the point of the problem!)  How does this relate to Problem
\ref{exponentialpp1}?\label{exponentialpp2}
 \solution{
${n\choose i}a_ib_j$.  This shows that the coefficient of $x^n\over n!$
in the EGF for painting poles with four colors is the coefficient of
$x^n\over n!$ in the product of the EGF for painting poles with two
colors and the EGF for painting poles with two colors.
 }
\ep

Problem \ref{exponentialpp2} shows that the formula you got for the
coefficient of $x^n\over n!$ in the product of two EGFs is the formula we
get by splitting a set $N$ of poles into two parts and painting the poles
in the first part with red and white and the poles in the second part with
blue and yellow.  More generally, you could interpret your result in
Problem \ref{exponentialpp1} to say that the coefficient of $x^n\over
n!$ in the product $\sum_{i=0}^\infty a_i {x^i\over i!}
\sum_{j=0}^\infty b_j{x^j\over j!}$ of two EGFs  is the sum, over all
ways of splitting a set
$N$ of size
$n$ into an ordered pair of disjoint sets $I$ and $J$, of the product
$a_{|I|}b_{|J|}$.   In contrast, when we multiplied ordinary generating
functions, the coefficient of
$x^n$ in
$\sum_{i=0}^\infty a_i x^i
\sum_{j=0}^\infty b_j x^j$ is the sum of all $a_ib_j$ over  ordered
pairs of integers $i$ and $j$ with $i + j=n$.  In our combinatorial
interpretation of the product of two ordinary generating functions, we
had two sets $S$ and $T$ of objects, $a_i$ was the number of objects in
$S$ of value $i$, $b_j$ was the number of objects in $T$ of value $j$, and
$i+j$ was the total value of an ordered pair of objects, one from $S$ and
one from $T$.  In painting poles for streetlights, what was important was
not only the number of poles we selected to paint red and white, but the
actual {\em set} of poles we selected to paint red and white.  This
suggests that the value of a selection in this context ought to be a set
instead of simply an integer, even though the size of the set will still
play a role.  For example, the number of ways of paint the poles in a set
red and white depends only on the size of the set, and not which set we
choose, but in forming the product of the exponential generating
functions, we have to analyze all ways of choosing a set of that size. 
This suggests that to get a combinatorial interpretation of the product of
EGFs we should consider a different kind of value function, one whose
values are sets and not integers.  Thus what we want to consider is a
``set-valued value function."  Since this is a mouthful to say, we will
shorten it in the definition that follows.  We define a {\em set-value
function} from  a set
$S$ to a set
$Y$ to be a function
$V$ from $S$ to the set of subsets of $Y$ such that for all subsets $I$ of
$Y$ of the same size, the number of elements $s$ of $S$ whose value
$V(s)$ is $I$ is the same.  For example if $S$ is all ways of painting
some of the street light poles on the north side on Main Street using red
and white, and if
$V(s)$ is the set of poles actually painted red or white, then the number
of ways to paint some of the poles red and white  depends on the size of
the set of poles being painted, so for each set of poles of a given size,
the number of ways to paint that set of poles using red and white is the
same.

\bp 
\itemes In Problem \ref{exponentialbookshelf}, why is the set of books
that we actually put onto a shelf a set-value function on the set of
all ways to put some of the books on that shelf? 
 \solution{Because the number of ways to put a set $S$ of books onto a
shelf is the same (namely $|S|!$) for all sets $S$ of the same size.
 } 

\itemes In Problem \ref{telephonenetwork}, why is the set of people
actually using their phones a set-value function (assume nobody is
calling outside the telephone network)? Equivalently, given an
involution, that is, a permutation that is a product of two cycles and one
cycles, why is the set of elements that are actually in two-cycles a
set-value function?  Why is the set of people who are not using their
phones (or, equivalently, the set of elements in a product of two-cycles
and one-cycles that are in one-cycles) a set-value function?
 \solution{
Because the number of ways to break a given set of $2n$ people into
two-cycles depends only on $n$ and not the particular set of $2n$ people
we choose.  The number of ways to break up a set of size $n$ into
one-cycles is one, so it doesn't depend on which set of size $n$ we are
breaking up.  (In fact it doesn't depend on $n$ either, but that is
irrelevant here.)
 }

\itemesi If $S$ is a set of objects with $V$  a set-value function from
$S$ to some set $Y$ and $T$ is a set of objects with $U$ a set-value
function from $T$ to the same set $Y$, then what is the relationship
among the EGF for the number
$a_i$ of elements of $S$ whose set-value is any one particular  set of
size
$i$, the EGF for the number $b_j$ of elements of $T$ whose set-value is
any one particular set of size
$j$ and the EGF for the number $c_n$ of ordered pairs $(s,t)$ in $S\times
T$ such that the set values of $s$ and $t$ are disjoint sets whose union
is any one particular set of size $n$?\label{exponentialpp3}
 \solution{
The EGF for $a_i$ times the EGF for $b_j$ is the EGF for $c_n$.  Stated
as a theorem, 
 \begin{solutiontheorem} 
If $S$ is a set of objects with $V$  a set-value function from $S$
to some set $Y$ and the EGF for the number $a_i$ of elements of $S$ whose
$V$-value is any particular set $N$ of size $n$ is $f(x)=\sum_{n=0}^\infty
a_n{x^n\over n!}
$ and
$T$ is a set of objects with
$U$ a set-value function from $T$ to the same set $Y$ and the EGF for the
number of elements $b_n$ of $T$ whose $U$-value is any particular set $N$
of size $n$ is $g(x) = \sum_{i=0}^\infty b_n{x^n\over n!}$, then  the EGF
for
 the number $c_n$ of ordered pairs $(s,t)$ in
$S\times T$ such that the set values of $s$ and $t$ are disjoint sets
whose union is any one particular set of size $n$ is $f(x)g(x)$.  That
is, the number of ordered pairs of the type described is the coefficient
of $x^n\over n!$ in $f(x)g(x)$.
\end{solutiontheorem}  The proof consists of interpreting Problem
\ref{exponentialpp1} in this context.
 } 

\ep
The theorem you proved in Problem \ref{exponentialpp3} is called the {\bf
product principle for exponential generating functions}\index{product
principle for exponential generating functions}\index{exponential
generating functions!product principle for}\index{generating
function!exponential!product principle for}.



\bp
\itemes Use the product principle for EGFs to explain the results of
Problems \ref{2cyclesand1cycles} and \ref{derangementsand1cycles}. 
 \solution{Every involution is a unique product of disjoint two-cycles and
one-cycles, so we can think of it as an ordered pair of whose first entry
is a set of disjoint two-cycles and whose second element is a set of
disjoint one-cycles.  The value of a set of two-cycles is the union of
their support sets and the value of a set of one-cycles is the union of
their support sets.  Then the EGF for involutions is the product of the
EGF for permutations that are the product of only disjoint two-cycles and
the EGF for permutations that are the product of only disjoint one-cycles
(i.e. identity permutations).  

We noted in the solution to Problem \ref{derangementsand1cycles} that
every permutation is the product of a derangement (on the elements that
are not fixed by the permutation) and a product of one cycles (on the
elements that are fixed by the permutation).  Thus we can think of every
permutation as an ordered pair consisting of a derangement and a product
of one-cycles, and the product principle tells us that the EGF for all
permutations is the product of the EGF for derangements and the EGF for
identity permutations.
 }



\ep

The product principle for EGFs has a natural extension to a product of
some arbitrary number $k$ of exponential generating functions.  Instead
of dealing with ordered pairs, it deals with (ordered) $k$-tuples.  Since
it is inconvenient to state unless one is careful with notation, we will
state it here.  The proof would be quite similar to your proof in Problem
\ref{exponentialpp3}.

\begin{theorem} If $f_1(x)$, $f_2(x)$, \ldots, $f_k(x)$ are the
exponential generating functions for sets $S_1$, $S_2$, \ldots, $S_k$
according to the set-value functions $V_1$, $V_2$, \ldots, $V_k$, then
$f_1(x)f_2(x)\cdots f_k(x)$ is the exponential generating function for the
number
$c_n$ of (ordered) $k$-tuples $(s_1,s_2,\ldots,s_k)$ with $s_i\in S_i$
such that
$V_1(s_1)$, $V_2(s_2)$, \ldots, $V_k(s_k)$ are disjoint sets whose union
is any one particular set $N$ of size $n$.
\end{theorem}

\begin{corollary} If $f(x)$ is the exponential generating function for
a set $S$ according to the set-value function $V$, then $f(x)^k$ is the
exponential generating function in which $a_n$ is the number of
$k$-tuples of elements of $S$ whose values are disjoint sets whose union
is any particular set of size $n$.\label{EGFtothen}\end{corollary}

\bp 

\itemes Use the general product principle for EGFs or its corollary to
explain the relationship between the EGF for painting streetlight poles
in only one color and the EGF for painting streetlight poles in $4$
colors in Problems \ref{paintinglightpoles} and \ref{paintinglightpoles2}.
What is the generating function
for the number $p_n$ of ways to paint  $n$ streetlight poles
with some fixed number $k$ of colors of paint.
 \solution{
We can think of a painting of a set of street poles as a four-tuple of
sets, the sets painted each of the four colors.  Then Corollary
\ref{EGFtothen} tells us that the EGF for such four-tuples is the fourth
power of the EGF for the number of ways to paint streetlight poles with
one color.  The EGF for painting streetlight poles with $k$ colors of
paint is $e^{kx}$.} 

\itemes Use the general product principle for EGFs or its corollary to
explain the relationship between the EGF for arranging books on one shelf
and the EGF for arranging books on
$n$ shelves in Problem
\ref{exponentialbookshelf}.
 \solution{
An arrangement of books on $n$ shelves may be thought of as a $n$-tuple
of arrangements of books on one shelf, and the value of an arrangement
may be taken to be the set of books actually on the shelf (or shelves). 
The Corollary \ref{EGFtothen} tells us that the EGF is the $n$th power of
the EGF for arranging books on one shelf, which is the EGF for
permutations.  Thus the EGF for arranging books on $n$ shelves is
$(1-x)^{-n}$.}


\itemi (Optional) Our very first example of exponential generating
functions used the binomial theorem to show that the EGF for $k$-element
permutations of an $n$ element set is $(1+x)^n$.  Use the EGF for
$k$-element permutations of a one-element set and the product principle
to prove the same thing. Hint: Review the alternate definition of a
function in Section \ref{orderedfunctionsection}.
 \solution{In Section \ref{orderedfunctionsection} we remarked that an
alternate definition of a function from $S$ to $T$ is that it is an
assignment of disjoint subsets of $S$ to elements of $T$ so that the
union of the subsets is $S$.  Thus a function from $[k]$ to $[n]$ may be
thought of as an $n$-tuple of disjoint subsets of $S$ whose union is
$[k]$.  In  particular, an injection from $[k]$ to $[n]$ (which is a
$k$-element permutation of $[n]$) can be thought of as a $n$-tuple of
disjoint singleton sets and empty sets whose union is $n$.  The number of
such $n$-tuples is therefore the number of $k$-element permutations of
$[n]$.  If $n=1$, the possible $n$-tuples are $(\emptyset)$ and $\{1\}$,
and so the EGF for such $n$-tuples is $1+x$.  Then by
Corollary \ref{EGFtothen}, if we let the value of an $n$-tuple of
disjoint singletons and empty sets be the union of the sets in the
$n$-tuple, the EGF for the number of $n$-tuples whose value is $[k]$ is
$(1+x)^n$.  Thus this is the EGF for $k$-element permutations of
$[n]$.}

\item What is the EGF for the number of ways to paint $n$
streetlight poles red, white blue, green and yellow, assuming an even
number of poles must be painted green and an even number of poles must be
 painted yellow?
\solution{By the product principle for exponential generating functions it
is 
$$e^{2x}\left(1+{(2x)^2\over 2!}+ {(2x)^4\over 4!} + \cdots \right) =
e^{2x}{e^{2x}-e^{-2x}\over2}=e^{2x}\cosh(2x). $$}

\itemesi What is the EGF for the number of functions from an
$n$-element set onto a one-element set?  (Can there be any functions
from the empty set onto a one-element set?)  What is the EGF for the
number
$c_n$ of functions from an $n$-element set onto a
$k$ element set (where $k$ is fixed)?  Use this EGF to find an explicit
expression for the number of functions from a $k$-element set onto an
$n$-element set and compare the result with what you got by inclusion and
exclusion.
 \solution{
There are no onto functions from a $0$-element set to a
$1$-element set; otherwise there is exactly one function from an
$n$-element set onto a one element set so the EGF for functions from an
$n$-element set onto a one element set is $e^x-1$.  A function from an
$n$-element set onto a $k$-element set may be thought of as a $k-tuple$
of functions from disjoint subsets whose union is the $n$-element set
onto the one element subsets of the $k$-element set.  Therefore by the
corollary to the product principle for EGFs the EGF for functions from an
$n$-element set onto a $k$-element set is $(e^x-1)^k$.  By the binomial
theorem, this is $$\kern-.25 in\sum_{i=0}^k{k\choose i}(-1)^{k-i}e^{ix} 
=
\sum_{i=0}^k {k\choose i} (-1)^{k-i} \sum_{j=0}^\infty {(ix)^j\over j!} = 
\sum_{i=0}^k (-1)^{k-i}{k\choose i} \sum_{j=0}^\infty i^j{(x)^j\over
j!}.$$
Thus $c_n= \sum_{i=0}^k(-1)^{k-i}{k\choose i}n^i$, which is consistent
with the formula we got by inclusion and exclusion.}

\itemesi In Problem \ref{BellNumberIntro} You showed that the Bell Numbers
$B_n$ satisfy the equation
$B_{n+1} =
\sum_{k=0}^{n} {n\choose k}B_{n-k}$ (or a similar equation for $B_n$.)
Multiply both sides of this equation by $x^n\over n!$ and sum from $n=0$
to infinity.  On the left hand side you have a derivative of a certain
EGF we might call $B(x)$.  On the right hand side, you
have a product of two EGFs, one of which is $B(x)$.  What is the other
one?  What differential equation involving $B(x)$ does this give you. 
Solve the differential equation for $B(x)$.  This is the EGF for the
Bell numbers!.  \label{BellNumbersEGF}
 \solution{
$$\sum_{n=0}^\infty B_{n+1}{x^n\over n!} = \sum_{n=0}^\infty{n\choose
k}B_{n-k}{x^n\over n!} =\sum_{i=0}^\infty B_{i}{x^{i}\over
(i)!}\sum_{j=0}^\infty {x^j\over j!}.$$
Thus $B'(x) = B(x)e^x$, which gives us $\ln B(x) = e^x+c$, or $B(x) =
e^{ex+c} =e^ce^{(e^x)}$.  Since $B_0=B(0)$ and $B_0=1$, we have $c=-1$ and
$B(x) = \exp(e^x-1)$.}

 \itemi Prove that  $n2^{n-1} = \sum_{k=1}^n {n\choose k}k$ by using
EGFs.
\solution{By the product principle for EGFs, the EGF for the right hand
side is $$e^x\cdot xe^x =xe^{2x}=x\sum_{i=o}^\infty {(2x)^i\over i!} =
\sum_{j=1}^\infty 2^{j-1}{x^j\over (j-1)!}=\sum_{j=1}^\infty
j2^{j-1}{x^j\over j!}.$$  Thus the coefficient of $x^n\over n!$ is
$n2^{n-1}$, as well as $\sum_{k=1}^n {n\choose k}k$.}

\itemes In light of Problem \ref{oneblockpartitions}, why is the EGF for
the Stirling numbers
$S(n,k)$ of the second kind not
$(e^x -1)^n$?  What is it equal to instead?
 \solution{Notice that a one block partition is the same thing as a
function from that block onto a one element set.  However a partition
with $n$ blocks is not an $n$-tuple of blocks, but rather a set of $n$
blocks.  An $n$-tuple of blocks corresponds to a function from the union
of the blocks onto an $n$-element set, and $n!$ different onto functions
correspond to the same partition into $n$ blocks.  Thus the EGF for
partitions of an $n$-element set into $k$ parts (where $n$ is fixed but
$k$ varies) is ${1\over n!} (e^x -1)^n$.
 }
\ep

The idea of the set-value function in the product principle for
exponential generating functions helps to resolve a mystery that you may
have been consciously or unconsciously wondering about.  How do we know
when ordinary generating functions will be most useful and how do we know
when exponential generating functions will be most useful?  When we are
in a situation---such as distributing fruit to children, or partitioning
an integer---where a combinatorial structure is determined by the number
of objects of a certain type (for example a partition of an integer is
determined by the number of ones, the number of twos and so on), ordinary
generating functions are most likely to be useful.  However when what
determines a combinatorial structure is the set of objects, or some
structure (such as a permutation) on a set of objects, then exponential
generating functions are most likely to be useful.  In particular, in
situations where our structure comes with labels, then exponential
generating functions are likely to be useful.  However there is a grey
area.  When we are distributing identical candy to children, the children
are labelled (they have names), but the candy is not.  Experience tells
us, though that since the candy is not labelled, ordinary generating
functions are useful.  So while the question of whether the most natural
value function seems to be an integer value or a set-value is a good
guideline, in the end experience helps tremendously! 

\section{The Exponential Formula}
Exponential generating functions turn out to be quite useful in advanced
work in combinatorics.  One reason why is that it is often possible to
give a combinatorial interpretation to the composition of two exponential
generating functions. In particular, if
$f(x) =
\sum_{i=0}^n a_i{x^i\over i!}$ and
$g(x) = \sum_{j=1}^\infty b_j {x^j\over j!}$, it makes sense to form the
composition $f(g(x))$ because in so doing we need add together only
finitely many terms in order to find the coefficient of $x^n\over n!$ in
$f(g(x))$ since in the EGF  $g(x)$ the dummy variable $j$ starts at 1.
Since our study of combinatorial structures has not been advanced enough
to give us applications of a general formula for the composition of the
EGF, we will not give here the combinatorial interpretation of this
composition.  However we have seen some examples where one particular
composition can be applied.  Namely, if $f(x) = e^x = \exp(x)$, then
$f(g(x)) =\ exp(g(x))$ is well defined when $b_0=0$.  We have seen three
examples in which an EGF is $e^{f(x)}$
where $f(x)$ is another EGF.  There is a fourth example in which the
exponential function is slightly hidden.

\bp
\itemes If $f(x)$ is the EGF for the number of partitions of an $n$-set
into one block, and $g(x)$ is the EGF for the total number of partitions
of an
$n$-element set, that is, for the Bell numbers $B_n$, how are the two
generating functions related?  \label{exp(oneblock)}
 \solution{
The EGF for one-block partitions is $e^x-1$ and for the Bell
numbers is $\exp(e^x-1)$, and so the EGF for the Bell numbers is
composition of the exponential function with the EGF for one block
partitions.
 }

\itemes Let  $f(x)$ be the EGF for the number of
permutations of an $n$-element set with one cycle of size one or two. 
What is
$f(x)$?  What is the EGF $g(x)$ for the number of permutations of an
$n$-element set all of whose cycles have size one or two, that is, the
number of involutions in $S_n$?  How are these two exponential generating
functions related?\label{exp(oneortwo-cycle)}
 \solution{
There is one permutation with one cycle of size 1, and one
permutation with one cycle of size 2.  Therefore the EGF for such
permutations is $x+{x^2\over 2!}= x+x^2/2$.  The EGF for involutions
is $e^{x+x^2/2}$.  Thus $g(x) = \exp(f(x))$.
 }

\itemesi Let $f(x)$ be the EGF for the number of permutations of an
$n$-element set that have exactly one two-cycle and no other cycles. 
Let
$g(x)$ be the EGF for the number of permutations which are products of
two-cycles only, that is, for tennis pairings.  (That is, they are not
a product of two-cycles and a nonzero number of one-cycles).  What is
$f(x)$?  What is
$g(x)$?  How are these to exponential generating functions
related?\label{exp(two-cycle)}
 \solution{
The EGF $f(x)$ for permutations of an $n$ element set that have exactly
one two-cycle (and no other cycles) is $x^2\over 2!$ or $x^2/2$.  By
Problem
\ref{exponentialtennisparings}, the EGF to permutations that are products
of two-cycles only is $\exp(x^2/2)$.  Thus $g(x)=\exp(f(x))=e^{x^2/2}$.
 }

\itemes Let $f(x)$ be the EGF for the number of
permutations of an $n$-element set that have exactly one cycle.  (This is
the same as the EGF for the number of ways to arrange $n$ people around a
round table.)  Let $g(x)$ be the EGF for the total number of permutations
of an $n$-element set.  What is $f(x)$?  What is $g(x)$?  How are $f(x)$
and $g(x)$ related?\label{exp(onecycle)}
 \solution{
In Problem \ref{exponentialroundtable} we showed that $f(x) =
\ln\left({1\over 1-x}\right)$.  In Problem \ref{exponentialpermutations}
we showed that $g(x)={1\over 1-x}$.  Therefore $g(x)= \exp(f(x))$.}
\ep

There was one element that our last four problems had in common.  In each
case our EGF $f(x)$ involved the number of structures of
a certain type (partitions, telephone networks, tennis pairings,
permutations) that used only one set of an appropriate kind.  (That is, we
had a partition with one part, a telephone network consisting either of
one person or two people connected to each other, a tennis pairing of one
set of two people, or a permutation with one cycle.)  Our EGF $g(x)$ was
the number of structures of the same ``type" (we put type in quotation
marks here because we don't plan to define it formally) that could consist
of any number of sets of the appropriate kind.  Notice that the order of
these sets was irrelevant.  For example we don't order the blocks of a
partition and a product of disjoint cycles is the same no matter what
order we use to write down the product.  Thus we were relating the EGF for
structures which were somehow ``building blocks" to the EGF for structures
which were sets of building blocks. For a reason that you will see
later, it is common to call the building blocks {\em
connected}\index{connected structures and EGFs}\index{exponential
generating functions for connected structures} structures. Notice that
our connected structures were all based on nonempty sets, so we had no
connected structures whose value was the empty set.  Thus in each case,
if $f(x) = \sum_{i=0}^\infty a_i{x^i\over i!}$, we would have $a_0=0$.
The relationship between the EGFs was always
$g(x) = e^{f(x)}$.  We now give a combinatorial explanation for
this relationship.

\bp  

\itemes Suppose that $S$ is a set with a set-value function $V$ from $S$
to a set
$Y$ such that no element of $S$ has the empty set as a value.  Let $f(x)$
be the generating function for $S$ according to the value function
$V$.\label{exponentialformula}
\begin{enumerate}
\item In the power series $$e^{f(x)} = 1 + f(x) + {f(x)^2\over 2!} +
\cdots + {f(x)^k\over k!} + \cdots= \sum_{k=0}^\infty {f(x)^k\over k!},$$
the product principle tells us that the coefficient of $x^n\over n!$ in
$f(x)^k$ is the number of $k$-tuples of elements of $S$ whose
values are disjoint sets whose union is any one particular subset $N$ of
$Y$ of size $n$.  How do you know that all of the elements of the
$k$-tuple are different?
 \solution{
Because the coefficient of $x^0$ is 0, there are no elements in $S$ whose
value is the empty set.
 }
\item When you divide the coefficient of $x^n\over n!$ in $f(x)^k$ by
$k!$, it no longer counts $k$-tuples whose values are disjoint sets whose
union is that set $N$.  What does it count instead?  (Hint: how many
$k$-tuples can you form with a set of $k$ distinct elements?)
 \solution{It counts $k$-element subsets since the entries of the
$k$-tuples are distinct.
 }
\item Why does this prove that the coefficient of $x^n\over n!$ in
$e^{f(x)}$ is the number of subsets of $S$ such that the values of th
elements of the subset partition any one particular set $N\subseteq T$ of
size
$n$?
\solution{We know that the coefficient of $x^n\over n!$ in $e^{f(x)}$ is
the sum over all $k$ of the number of $k$-element subsets of $S$ whose
values partition a particular set $N$.  (The values must partition $N$
because none of them are empty, they are disjoint, and their union is
$N$.)  This sum is the number of subsets of $S$ such that the values of
the elements of the subset partition a particular set $N$ of size $n$.
}
\end{enumerate}
\ep
In Problem \ref{exponentialformula} we proved the following theorem,
which is called the {\bf exponential formula}.\index{exponential formula}

\begin{theorem}  Suppose that $S$ is a set with a set-value function $V$ from it to a
set
$Y$ such that no element of $S$ has the empty set as a value.  Let $f(x)$
be the generating function for $S$ according to the value function
$V$.  Then the coefficient of $x^n\over n!$ in
$e^{f(x)}$ is the number of subsets of $S$ such that the values of the
elements of the subset partition any one particular set $N\subseteq T$ of
size
$n$.
\end{theorem}

Since the statement of the theorem is rather abstract, let us see how it
applies to the examples in Problems \ref{exp(oneblock)},
\ref{exp(oneortwo-cycle)}, \ref{exp(two-cycle)} and \ref{exp(onecycle)}. 
In Problem \ref{oneblockpartitions} our set $S$ should consist of
one-block partitions of sets; that is, it should be a set of nonempty
sets.  The value of a one-block partition will be the set it partitions,
and we want every subset (of a given size) of our set $Y$ to be the value
of the same number of partitions.  Thus we can take $S$ to be the set of
all nonempty finite subsets of the positive integers and take $Y$ to be
the set of positive integers.  Since a partition of a set is a set of
blocks whose union is
$S$, a one block partition whose block is $B$ is the set $\{B\}$.  We take
$V(\{B\}) =B$.  Then any nonempty finite subset of of the positive
integers is the value of exactly one member of $S$, and the empty set and
all infinite subsets of the positive integers are the value of exactly
zero members of $S$.  Thus we have a value function.  As you showed in
Problem 
\ref{oneblockpartitions} the generating function for partitions with just
one block is $e^x-1$.  Thus by the exponential formula, $\exp(e^x-1)$ is
the generating function for sets of subsets of the positive integers
whose values are disjoint sets whose union is any particular set $N$ of
size $n$.  Since the values are nonempty sets, this means they partition
the set $N$.  Thus $\exp(e^x-1)$ is the generating function for partitions
of sets of size $n$.  (Technically, it is the generating function for
partitions of subsets of the integers of size $n$, but any two
$n$-element sets have the same number of partitions.)

\bp 
\itemes Explain how the exponential formula proves the relationship we saw
in Problem \ref{exp(onecycle)}.
\solution{We take $S$ to be the set of permutations of sets of positive
integers that have just one-cycle.  We take the value of a one-cycle
permutation to be the support set of the cycle.  The number of one-cycle
permutations whose value is any given $n$-element set of positive
integers is $(n-1)!$, so we have defined a value function.  We take the
set
$Y$ to be the set of positive integers.  Then by the Exponential Formula,
if
$f(x)$ is the EGF for permutations with one cycle, $\exp(f(x))=g(x)$ is
the EGF in which the coefficient of $x^n\over n!$ is the number of sets of
permutations with one cycle whose support sets partition any given set $N$
of size
$n$.  That is, the coefficient of $x^n\over n!$ in $g(x)$ is the the
number of products of disjoint cycles which partition any given set $N$;
that is, it is the EGF for permutations of
$N$.
}

\itemes Explain how the exponential formula proves the relationship we saw
in Problem \ref{exp(two-cycle)}.
 \solution{We let $S$ be the set of permutations of sets of positive
integers that consist of exactly one two-cycle. We take the set $Y$
to be the set of positive integers.  We take the value of a
two-cycle to be its support set.  Since the number of two-cycles with a
given support set is one, this is a value function.  We saw in Problem
\ref{exp(two-cycle)} that the EGF for
$S$ with this value function is $x^2/2$. By the exponential formula, the
EGF for sets of disjoint two-cycles is $\exp(x^2/2)$.  But a sets of
disjoint two-cycles correspond bijectively with permutations of subsets
of the positive integers which are products of disjoint two-cycles
(only), and this confirms the result of Problem
\ref{exponentialtennisparings}.
}

\itemes Explain how the exponential formula proves the relationship we saw
in Problem \ref{exp(oneortwo-cycle)}.
\solution{
We let the set $S$ be the set of permutations of subsets of the
positive integers that consist of a one-cycle or consist of a two-cycle.
We let the set
$Y$ be the set of positive integers.  We let the value of a permutation
in $S$ be the support set of its cycle.  We saw in problem
\ref{exp(oneortwo-cycle)} that the EGF for $S$ is $x+x^2/2$.  By the
exponential formula, the EGF for sets of disjoint one and two-cycles is
$e^x+x^2/2$.  But there is a bijection between the sets of disjoint one
and two-cycles and permutations that are a product of disjoint one and
two-cycles.  This confirms the result of Problem
\ref{telephonenetworkEGF}.
}

\itemes In Problem \ref{paintinglightpoles} we saw that the generating
function for the number of ways to use four colors of paint to paint $n$
light poles along the north side of Main Street in Anytown was $e^{4x}$. 
We should expect an explanation of this generating function using the
exponential formula.  Let $S$ be the set of all ordered pairs consisting
of a light pole and a color.  Thus a given light pole occurs in four
ordered pairs.  What is a natural set-value function on $S$?  What is the
exponential generating function $f(x)$ for $S$ according to that value? 
Assuming that there is no upper limit on the number of light poles, what
subsets of $S$ does the exponential formula 
tell us are counted by the coefficient of $x^n$ in $e^{f(x)}$?  How do
the sets being counted relate to ways to paint light poles?
 \solution{
A natural set-value function on $S$ assigns to each ordered pair the
singleton set consisting of the light pole in the pair.  The EGF for $S$
according to that value is $4x$, because there are four ordered pairs
with any given value.  Note that a set of ordered pairs whose values
partition a set $N$ of light poles $N$ is exactly a function from the set
$N$ of light poles to the set of colors.  Then by the exponential
formula, the generating function for the number of ways to paint $n$
light poles with four colors is $e^{4x}$.
}

\ep





One of the most spectacular applications of the exponential formula
is also the reason why, when we regard a combinatorial structure as a set
of building block structures, we call the building block structures {\em
connected}.  In Chapter 2 we introduced the idea of a connected graph and
in Problem
\ref{connectedanddisconnected} we saw examples of graphs which were
connected and were not connected.  A subset $C$ of the vertex set of a
graph is called a {\bf connected component}\index{connected component
of a graph}\index{graph!connected component of} of the graph if
\begin{itemize}
\item every vertex in $C$ is
connected to every other vertex in that set by a walk whose vertices lie
in $C$, and
\item no other vertex in the graph is connected by a walk to any vertex
in $C$.
\end{itemize}
In Problem \ref{conncomp} we showed that each connected component of a
graph consists of a vertex and all vertices connected to it by walks in
the graph.

\bp
\itemes Show that every vertex of a graph lies in one and only one
connected component of a graph.  (Notice that this shows that the
connected components of a graph form a partition of the vertex set of the
graph.)
\solution{
Let $C$ be the set of all vertices connected by a walk to a vertex $x$. 
Then
\begin{itemize}
\item Each pair of vertices $u$ and $v$ in $C$ is connected by the walk
that goes from $u$ to $x$ and then from $x$ to $v$.
\item If a vertex $w$ is connected by a walk to a vertex $v$ in $C$, then
it is connected to $x$ by the walk that goes from $w$ to $v$ and then
from $v$ to $x$.  Thus no  vertex $w$ in the graph other than a member of
$C$ is connected by a walk to any vertex in $C$.
\end{itemize}
Therefore $C$ is a connected component containing $x$.  If 
connected component $D$ contained $x$, then every vertex in $D$ would be
connected by a walk to $x$ and then by a walk from $x$ to $v$ for each
other vertex in $C$, and similarly, each vertex in $C$ would be connected
to each vertex in $D$.  Thus by the definition of connected component,
$C$ and $D$ would have to be the same set.  Therefore each vertex lies in
one and only one connected component.
}


\itemes Explain why no edge of the graph connects two vertices in
different connected components.
\solution{If an edge connected two vertices in different connected
components, that edge would give a walk from a vertex in one of the
connected components to a vertex in the other connected component, and
thus not in the first component, violating the second part of the
definition of a connected component.
}

\itemes  Explain why it is that if $C$ is a connected component of a
graph and
$E'$ is the set of all edges of the graph that connect vertices in $C$,
then the graph with vertex set $C$ and edge set
$E'$ is a connected graph.  We call this graph a {\em connected component
graph}\index{connected component graph} of the original graph.
\solution{
If there is a walk between two vertices in a connected component, all
edges of the walk must connect two vertices in the connected component,
because if there were an edge in the walk that did not do so, it would
violate the second part of the definition of a connected component.
}

\ep
The last sequence of problems shows that we may think of any graph as the
set of its connected component graphs.  (Once we know them, we  know all
the vertices and all the edges of the graph).  Notice that a graph is
connected if and only if it has exactly one connected component.  Since
the connected components form a partition of the vertex set of a graph,
the exponential formula will relate the generating function for the
number of connected graphs on $n$ vertices with the generating function
for the number of graphs (connected or not) on $n$ vertices. However
because we can draw as many edges as we want between two vertices of a
graph, there are infinitely many graphs on $n$ vertices, and so the
problem of counting them is uninteresting.  We can make it interesting by
considering {\bf simple graphs},\index{graph!simple}\index{simple graph}
namely graphs in which each edge has two distinct endpoints and no two
edges connect the same two vertices. It is because connected graphs form
the building blocks for viewing all graphs as sets of connected components
that we refer to the building blocks for structures counted by the
generating functions in the exponential formula as connected
structures.\index{exponential formula!connected structures for}

\bp
\itemesi Suppose that $f(x) = \sum_{n=0}^\infty c_n {x^n\over n!}$ is
the exponential generating function for the number of simple
connected graphs on
$n$ vertices and $g(x) = \sum_{i=0}^\infty a_i {x^i\over i!}$ is the
exponential generating function for the number of simple graphs on $i$
vertices.  
\begin{enumerate}
\item Is $f(x) = e^{g(x)}$, is $f(x) = e^{g(x)-1}$, is $g(x) = e^{f(x)-1}$
or is
$g(x) = e^{f(x)}$?
\solution{To apply the exponential formula, we must take the exponential
function of an EGF whose constant term is zero, or in other words for a
set of elements whose values are nonempty sets.  We can let $S$ be the
set of nonempty connected graphs. (Technically, the graph with the empty
set of vertices and the empty set of edges is connected.)  Therefore
$f(x) -1$ is the EGF for $S$. By the exponential formula,
$g(x)=e^{f(x)-1}$ because a simple graph may be thought of as a set of
simple connected graphs, namely its connected component graphs. (Note that
$g(x)$ has 1 for its constant term, which corresponds to thinking of the
empty graph as having an empty set of nonempty connected components.)
}
\item One of $a_i$ and $c_n$ can be computed by recognizing that a
simple graph on a vertex set $V$ is completely determined by its edge set
and its edge set is a subset of the set of two element subsets of $V$. 
Figure out which it is and compute it.
\solution{To specify a simple graph on a vertex set $V$, we have to
specify its set of edges.  The possible sets of edges thus correspond
bijectively to sets of two-element subsets of $V$.  But if $V$ has size
$i$ the set of all two element subsets of $V$ has $i\choose 2$ elements. 
Thus the number of sets of two element subsets of $V$ is $2^{i\choose
2}$.  Therefore $a_i = 2^{i\choose 2}$.
}
\item Write $g(x)$ in terms of the natural logarithm of $f(x)$
or $f(x)$ in terms of the natural logarithm of  $g(x)$.
\solution{
Since $g(x) = e^{f(x)-1}$, $f(x) = 1+ \ln g(x)$.
}

\item Write $\log(1+y)$ as a power series in $y$.
\solution{$$\log(1+y)=\int_0^y {1\over 1+x}dx =\int_0^y \sum_{i=0}^\infty
(-1)^ix^i = \sum_{j=1}^\infty (-1)^{j-1}{y^j\over j}.$$
}
\item   Why is the coefficient
of
$x^0\over 0!$ in $g(x)$ equal to one?  Write $f(x)$ as a power series in
$g(x) -1$.
\solution{The coefficient of $x^0\over 0!$ is 1 because there is one
graph on the empty set; the one with no edges.  $$f(x) = 1 +\ln(1 +
(g(x)-1))=1+\sum_{j=1}^\infty (-1)^{j-1}{(g(x)-1)^j\over j}.$$
}
\item You can now use the previous parts of the Problem to find a formula
for $c_n$ that involves summing over all partitions of the integer $n$. 
(It isn't the simplest formula in the world, and it isn't the easiest
formula in the world to figure out, but it is nonetheless a formula
with which one could actually make computations!)  Find such a  formula.
 \solution{
$$f(x) = 1+\sum_{j=1}^\infty
(-1)^{j-1}{(g(x)-1)^j\over j}= 1+\sum_{j=1}^\infty
(-1)^{j-1}{(\sum_{i=1}^\infty 2^{i\choose 2}{x^i\over i!})^j\over j}.$$
From the right-hand expression, we get a term involving $x^n$ whenever we
have an $x^n$ term in the $j$th power of $\sum_{i=1}^\infty 2^{i\choose
2}{x^i\over i!}$.  So the coefficient of
$x^n\over n!$ is the sum over all $j$ and all sequences
$i_1,i_2,\ldots,i_j$ with
$i_1+i_2+\cdots+i_j = n$ of terms of the form $$ {n!\over
j}(-1)^j\prod_{k=1}^j {2^{i_k\choose 2}
\over i_k!},$$ where  each $i_k>0$.  Notice that reordering the numbers
$i_1$, $i_2$, \ldots $i_k$ does not change the value of the expression. 
The sequence of $i_k$s is a composition of $n$ into positive parts.  If
we knew how many compositions of $n$ into $j$ parts correspond to one
partition of $n$ into $j$ parts, we could sum over a much smaller set of
terms.  If we use the type vector notation for a partition, namely that
it has $p_1$ parts of size 1, $p_2$ parts of size 2, \ldots, $p_n$ parts
of size $n$, then the number of compositions corresponding to that
partition, i.e. the number of compositions with the type vector
$(p_1,p_2, \ldots, p_n)$ is the number of ways to take $j$ places in a
vector and label $p_1$ of them with 1, $p_2$ of them with 2, and so on
until we label $p_n$ of them with $n$.  This number is the multinomial
coefficient $j\choose p_1,p_2,\ldots, p_n$.  Thus our sum over all j and
all compositions of $n$ into $j$ parts becomes
$$\sum_{j=1}^n {n!\over j}(-1)^j\sum_{{p_1,p_2,\ldots,p_n: \sum_{r=1}^n
rp_r =n \atop \mbox{and} \sum_{r=1}^n p_r =j}}{j\choose p_1, p_2, \ldots,
p_n}
\prod_{r=1}^n{2^{^{{r\choose 2}p_r}}
\over (r!)^{p_r}}.$$  We can remove one of the summation signs and the
condition that $\sum_{r=1}^n p_r=j$ by substituting $\sum_{r=1}^n p_r$
for $j$, and we get
$$\sum_{{p_1,p_2,\ldots,p_n: \sum_{r=1}^n
rp_r =n }}{n!}(-1)^{^{\sum_{r=1}^n p_r}}\left(\sum_{r=1}^n p_r -1\right)!
\prod_{r=1}^n{2^{^{{r\choose 2}p_r}}
\over (r!)^{p_r}p_r!}$$
for the number of connected graphs on $n$ vertices.  If we want to
shorten the appearance of the formula we can keep $j$ in our sum and
explain its value afterwards, as in
$$\sum_{{p_1,p_2,\ldots,p_n: \sum_{r=1}^n
rp_r =n }}{n!}(-1)^j(j-1)!
\prod_{r=1}^n{2^{^{{r\choose 2}p_r}}
\over (r!)^{p_r}p_r!},$$
where $j=\sum_{r=1}^n p_r$.
}
\end{enumerate}
\ep

The point to the last problem is that we can use the exponential formula
in reverse to say that if $g(x)$ is the generating function for the
number of (nonempty) connected structures of size $n$ in a given family of
combinatorial structures and $f(x)$ is the generating function for all
the structures of size $n$ in that family, then not only is $f(x) =
e^{g(x)}$, but $g(x) = \ln(f(x))$ as well.  Further, if we happen to have
a formula for either the coefficients of $f(x)$ or the coefficients of
$g(x)$, we can get a formula for the coefficients of the other one!


\section{Supplementary Problems}
\begin{enumerate} 

\item Use product principle for EGFs and the idea of coloring
a set in two colors to prove the formula $e^x\cdot e^x = e^{2x}.$

\item Find the EGF for the number of ordered functions from a $k$-element
set to an $n$-element set.

\item Find the EGF for the number of ways to string $n$ distinct beads
onto a necklace.

\item Find the exponential generating function for the number of broken
permutations of a $k$-element set into $n$ parts.

\item Find the EGF for the total number of broken permutations of a
$k$-element set.

\item Find the EGF for the number of graphs on $n$ vertices in which every
vertex has degree 2.

\item Recall that a cycle of a permutation cannot be empty.
\begin{enumerate}  
\item What is the generating function for the number of
cycles on an even number of elements (i.e. permutations of an even number
$n$ of elements that form an
$n$-cycle)?  Your answer should not have a summation sign in it.  Hint: 
If
$y=
\sum_{i=0}^\infty {x^{2i}\over 2i}$, what is the derivative of $y$?
\item What is the generating function for the number of permutations on
$n$ elements that are a product of even cycles?
\item What is the generating function for the number of cycles on an odd
number of elements?
\item What is the generating function for the number of permutations on
$n$ elements that are a product of odd cycles?
\item How do the generating functions in parts (b) and (d) of this
problem related to the generating function for all permutations on $n$
elements?

\end{enumerate}

\end{enumerate}

