\chapter{Applications of Induction and Recursion in Combinatorics and
Graph Theory}
\section{Some Examples of Mathematical Induction}
\markboth{CHAPTER 2.  APPLICATIONS OF INDUCTION IN COMBINATORICS}{SOME
EXAMPLES OF MATHEMATICAL INDUCTION} In Chapter 1 (Problem
\ref{SubsetsByInduction}), we used the principle of mathematical
induction to prove that a set of size $n$ has
$2^n$ subsets.  If you were unable to do that problem and you haven't yet
read Appendix \ref{Induction} (a portion of which is repeated here), you
should do so now.
\subsection{Mathematical induction}  The {\bf principle of mathematical
induction}\index{mathematical induction!principle of}\index{principle of
mathematical induction}\index{induction!mathematical, the principle of}
states that
\begin{quote}
In order to prove a statement about an integer $n$, if we
can\begin{enumerate}
\item Prove the statement when $n=b$, for some fixed integer $b$
\item Show that the truth of the statement for $n=k-1$ implies the truth
of the statement for $n=k$ whenever $k>b$,
\end{enumerate}
then we can conclude the statement is true for all integers $n\ge
b$.\end{quote} As an example, let us return to Problem
\ref{SubsetsByInduction}.  The statement we wish to prove is the
statement that ``A set of size $n$ has $2^n$ subsets."  

\begin{quote}Our statement
is true when $n=0$, because a set of size 0 is the empty set and the
empty set has $1=2^0$ subsets. (This step of our proof is called a {\em
base step}.)

Now suppose that $k>0$ and every set with $k-1$ elements has $2^{k-1}$
subsets.  Suppose $S=\{a_1,a_2,\ldots a_k\}$ is a set with $k$ elements. 
We partition the subsets of $S$ into two blocks.  Block $B_1$ consists of
the subsets that do not contain $a_n$ and block $B_2$ consists of the
subsets that do contain $a_n$.  Each set in $B_1$ is a subset of
$\{a_1,a_2,\ldots a_{k-1}\}$, and each subset of $\{a_1,a_2, \ldots
a_{k-1}\}$ is in $B_1$.  Thus $B_1$ is the set of all subsets of
$\{a_1,a_2,\ldots a_{k-1}\}$.  Therefore by our assumption in the first
sentence of this paragraph, the size of $B_1$ is $2^{k-1}$.  Consider the
function from $B_2$ to $B_1$ which takes a subset of $S$ including $a_k$
and removes $a_k$ from it.  This function is defined on $B_2$, because
every set in $B_2$ contains $a_k$.  This function is onto, because if
$T$ is a set in $B_1$, then $T\cup \{a_k\}$ is a set in $B_2$ which the
function sends to $T$.  This function is one-to-one because if $V$ and
$W$ are two different sets in $B_2$, then removing $a_k$ from them gives
two different sets in $B_1$.  Thus we have a bijection between $B_1$ and
$B_2$, so $B_1$ and $B_2$ have the same size.  Therefore by the sum
principle the size of
$B_1\cup B_2$ is $2^{k-1} +2^{k-1}=2^k$.  Therefore $S$ has $2^k$
subsets.  This shows that if a set of size $k-1$ has $2^{k-1}$ subsets,
then a set of size $k$ has $2^k$ subsets.  Therefore by the principle of
mathematical induction, a set of size $n$ has $2^n$ subsets for every
nonnegative integer $n$.
\end{quote}

The first sentence of the last paragraph is called the {\em inductive
hypothesis}.  In an inductive proof we always make an inductive
hypothesis as part of proving that the truth of our statement when
$n=k-1$ implies the truth of our statement when $n=k$.  The last
paragraph itself is called the {\em inductive step} of our proof.  In an
inductive step we derive the statement for $n=k$ from the statement for
$n=k-1$, thus proving that the truth of our statement when $n=k-1$
implies the truth of our statement when $n=k$.  The last sentence in the
last paragraph is called the {\em inductive conclusion}.  All inductive
proofs should have a base step, an inductive hypothesis, an inductive
step, and an inductive conclusion.  

There are a couple details worth noticing.  First, in this problem, our
base step was the case $n=0$, or in other words, we had $b=0$.  However,
in other proofs,
$b$ could be any integer, positive, negative, or 0.  Second, our proof
that the truth of our statement for $n=k-1$ implies the truth of our
statement for
$n=k$ required that $k$ be at least 1, so that there would be an element
$a_k$ we could take away in order to describe our bijection.  However,
condition (2) of the principle of mathematical induction only requires
that we be able to prove the implication for $k>0$, so we were allowed to
assume $k>0$.

\subsubsection{Strong Mathematical Induction}
One way of looking at the principle of mathematical induction is that it
tells us that if we know the ``first" case of a theorem and we can derive
each other case of the theorem from a smaller case, then the theorem is
true in all cases.  However the particular way in which we stated the
theorem is rather restrictive in that it requires us to derive each
case from the immediately preceding case.  This restriction is
not necessary, and removing it leads us to a more general statement of
the principal of mathematical induction which people often call the {\bf
strong principle of mathematical induction}.  It states:
\begin{quote}In order to prove a
statement about an integer $n$ if we can
\begin{enumerate}
\item prove our statement when $n=b$ and
\item prove that the statements we get with $n=b$, $n=b+1$, \ldots
$n=k-1$ imply the statement with $n=k$,
\end{enumerate}
then our statement is true for all integers $n\ge b$.
\end{quote}
You will find some explicit examples of the use of the strong principle of
mathematical induction in Appendix \ref{Induction} and will find some uses for
it in this chapter.

\subsection{Binomial Coefficients and the Binomial Theorem}
\bp
\iteme When we studied the Pascal Equation and subsets in Chapter 1, it
may have appeared that there is no connection between the Pascal relation
${n\choose k} = {n-1\choose k-1} +{n-1\choose k}$ and the formula
${n\choose k}={n!\over k!(n-k)!}$.  Of course you probably realize you
can prove the Pascal relation by substituting the values the formula
gives you into the right-hand side of the equation and simplifying to
give you the left hand side.  In fact, from the Pascal Relation and the
facts that ${n\choose 0}=1$ and ${n\choose n}=1$, you can actually prove
the formula for $n\choose k$ by induction on $n$.  Do so.
\solution{We wish to prove that ${n\choose i} ={n!\over i!(n-i)!}$.  We
note that since ${n\choose 0}=1$ and ${n\choose n} =1$ are in agreement
with this formula, we only have to consider the cases in which $0<i<n$,
which by the way, requires that $n\ge 2$.  We will prove that our formula
holds by induction on $n$ for $n\ge 2$.  If $n=2$, the only $i$ we need
to consider is $i=1$, and we know that ${2\choose1}=2$, the number of
one-element subsets of a two-element set.  But $2!\over1!(2-1)!$ is 2
also, so our formula holds when $n=2$.  Now suppose our formula holds
when $n=k-1$, so that for every $i$ with $0<i<n-1$, ${k-1\choose i} =
{(k-1)!\over i!(k-1-i)!}$.  Then by the Pascal Equation
\begin{eqnarray*}{k\choose i}&=&{k-1\choose i-1}+{k-1\choose i}\\ &=&
{(k-1)!\over (i-1)!(k-1-i+1)!} + {(k-1)!\over
i!(k-1-i)!}\\ &=&{(k-1)!i+(k-1)!(k-i)\over i!(k-i)!} \ =\  {k!\over
i!(k-i)!}.\end{eqnarray*}  Thus the truth of our formula for $n=k-1$
implies its truth for $n=k$.  Therefore by the principle of mathematical
induction, our formula is true for all integers $n\ge 2$.  We have
already seen is is true when $i=0$ or
$i=1$, so it is true for all nonnegative
$n$ and all numbers $i$ with $0\le i\le n$.}

\itemi Use the fact that $(x+y)^n = (x+y)(x+y)^{n-1}$ to give an inductive
proof of the binomial theorem.
\solution{We prove the binomial theorem by induction on $n$.  When $n=0$,
$(x+y)^n=(x+y)^0=1=\sum_{i=0}^0 {n\choose i}x^{0-i}y^i$ since that last
summation consists of the one term ${0\choose 0}x^0y^0$.

Now suppose that when $n=k-1$, $(x+y)^n=\sum_{i=0}^n {n\choose
i}x^{n-i}y^i.$  This gives us
\begin{eqnarray*}(x+y)^k&=&(x+y)(x+y)^{k-1}=(x+y)\sum_{i=0}^{k-1}{k-1\choose
i}x^{k-1-i}y^i\\
&=&\sum_{i=0}^{k-1}{k-1\choose i}x^{k-i}y^i+\sum_{i=0}^{k-1} {k-1\choose
i}x^{k-1-i}y^{i+1}\\
&=&\sum_{i=0}^{k-1}{k-1\choose i}x^{k-i}y^i+\sum_{i=1}^{k} {k-1\choose
i-1}x^{k-i}y^{i}\\
&=& x^k+\left(\sum_{i=1}^{k-1}{k-1\choose i}x^{k-i}y^i+{k-1\choose
i-1}x^{k-i}y^i\right) +y^k\\
&=& x^k+\left(\sum_{i=1}^{k-1} {k\choose i}x^{k-i}y^i\right) +y^k\ =\
\sum_{i=0}^k {k\choose i}x^{k-i}y^i.
\end{eqnarray*}Thus the truth of the binomial theorem for $n=k-1$ implies
its truth for $n=k$.  Then by the principle of mathematical induction, the
binomial theorem must be true for all integers $n\ge 0$.}

\item Suppose that $f$ is a function defined on the nonnegative integers
such that $f(0)=3$ and $f(n)=2f(n-1)$.  Find a formula for $f(n)$ and
prove your formula is correct.
\solution{$f(n)=3\cdot2^n$.  We prove our formula is correct by
induction.  When $n=0$ our formula gives $f(0)=3$, which is what we were
given.  Now suppose that when $n=k-1$, $f(n) =3\cdot2^n$.  Then
$f(k)=2\cdot  f(k-1) =2\cdot 3\cdot2^{k-1}=3\cdot2^k$.  Therefore the
truth of our formula when $n=k-1$ implies its truth when $n=k$ and so by
the principle of mathematical induction, $f(n)=3\cdot 2^n$ for all
nonnegative integers $n$.}
\ep
\subsection{Inductive definition}
You may have seen $n!$\index{factorial} described by the two equations
$0!=1$ and
$n!=n(n-1)!$ for $n>0$.  By the principle of mathematical induction we
know that this pair of equations defines $n!$ for all nonnegative numbers
$n$.  For this reason we call such a definition an {\bf inductive
definition}\index{inductive definition}\index{definition!inductive}.  An
inductive definition is sometimes called a {\em recursive
definition}\index{recursive definition}\index{definition!recursive}. 
Often we can get very easy proofs of useful facts by using inductive
definitions.
\bp 
\itemi An inductive definition of $a^n$ for nonnegative $n$ is given by
$a^0=1$ and $a^n=aa^{n-1}$.  (Notice the similarity to the inductive
definition of $n!$.)  We remarked above that inductive definitions often
give us easy proofs of useful facts.  Here we apply this inductive
definition to prove two useful facts about exponents that you have been
using almost since you learned the meaning of exponents.
\begin{enumerate}
\item Use this definition to prove the rule of
exponents $a^{m+n}=a^ma^n$ for nonnegative $m$ and $n$.
\solution{We use induction on $n$ to prove this.  When $n=0$, the formula
gives us
$a^{m+0} =a^ma^0=a^m\cdot 1=a^m$, so the rule of exponents holds when
$n=0$.  Now assume it holds when $n=k-1$ so that $a^{m+k-1}=a^ma^{k-1}$. 
Then, starting and ending with our inductive definition, we may write
$$a^{m+n}=aa^{m+n-1}=aa^ma^{k-1}=a^m\cdot a\cdot a^{k-1}=a^ma^k.$$  Thus
the truth of our law for $n=k-1$ implies its truth for $n=k$. 
Therefore, by the principle of mathematical induction, $a^{m+n}=a^ma^n$
for all nonnegative integers $n$.}

\item Use this definition to prove the rule of exponents $a^{mn} =
(a^m)^n$.
\solution{We will use induction on $n$ and part (a) of this problem to
prove that $a^{mn}=(a^m)^n$.  First, when $n=0$ the left and right hand
sides of the equation are both 1, so $a^{mn}=(a^m)^n$ holds when $n=0$. 
Now assume that $a^{m(k-1)} =(a^m)^{k-1}$.  This may be rewritten as
$a^{mk-m}=(a^m)^{k-1}$  Multiply both sides by $a^m$ and apply part (a)
of the problem and then the inductive definition (with $a^m$ replacing $a$)
to get
\begin{eqnarray*}
a^{mk-m}a^m&=&(a^m)^{k-1}a^m\\
a^{mk}&=&(a^m)^{k-1}a^m\\
a^{mk}&=&(a^m)^k.
\end{eqnarray*}
Thus the truth of our formula when $n=k-1$ implies its truth when $n=k$. 
Therefore by the principle of mathematical induction, the formula is true
for all nonnegative integers $n$.}
\end{enumerate}

\items Suppose that $f$ is a function on the nonnegative integers such
that $f(0)=0$ and $f(n) = n+f(n-1)$.  Prove that $f(n) = n(n+1)/2$.  Notice
that this gives a third proof that $1+2+\cdots+n=n(n+1)/2$, because this sum
satisfies the two conditions for $f$.  (The sum has no terms and is thus 0
when $n=0$.)
\solution{We prove the formula for $f$ by induction on $n$.  If $n=0$,
then $n(n+1)/2=0$ which is what we were given.  Now assume that $f(k-1)=
(k-1)k/2$.  Then $f(k)= k+f(k-1)= k+(k-1)k/2=(k^2+2k-k)/2=k(k+1)/2$.
Therefore the truth of the formula for $n=k-1$ implies its truth for
$n=k$, and thus by the principle of mathematical induction, the formula
for $f$ holds for all nonnegative integers $n$.}

\itemi Give an inductive definition of the summation notation
$\sum_{i=1}^n a_i$.  Use it and the distributive law $b(a+c) = ba+bc$ to
prove the distributive law
$$b\sum_{i=1}^n a_i = \sum_{i=1}^n ba_i.$$
\solution{
We define $\sum_{i=1}^1a_i = a_1$ and for $n>1$, $\sum_{i=1}^n
a_i =  \left(\sum_{i=1}^{n-1}a_i\right) +a_n$.  When $n=1$,
$b\sum_{i=1}^1a_i =ba_1$ by the base step of our inductive definition.
Assume that $k>1$ and $b\sum_{i=1}^{k-1}a_i=\sum_{i=1}^{k-1}ba_i$. Now we
can write
$$b\sum_{i=1}^k a_i\!=\! b\left[\left(\sum_{i=1}^{k-1}a_i\right)+a_k\right]
\!=\!
\left(b\sum_{i=1}^{k-1}a_i\right) +ba_k \!=\! \left(\sum_{i=1}^{k-1}ba_i\right)
+ ba_k \!=\! \sum_{i=1}^k ba_i,$$
where the last step is justified by the inductive step of our inductive
definition with $a_i$ replaced by $ba_i$.  Thus the truth of our
statement for $k-1$ implies its truth for $i=k$, and therefore by the
principle of mathematical induction, for all positive integers $n$, 
$b\sum_{i=1}^na_i= \sum_{i=1}^nba_i$.}

\ep


\subsection{Proving the general product principle (Optional)}
We stated the sum principle as 
\begin{quote}If we have a partition of a set $S$, then the size of $S$ is
the sum of the sizes of the blocks of the partition.\end{quote}
In fact, the simplest form of the sum principle says that the size of the
sum of two disjoint (finite) sets is the sum of their sizes.  
\bp 
\item Prove the sum principle we stated for partitions of a set from the
simplest form of the sum principle.
\solution{We prove by induction on $n$ that the size of the union of $n$
disjoint sets is the sum of their sizes.  We assume that the size of the
union of two disjoint sets is the sum of their sizes.  Now assume $k>2$
and the size of the union of $k-1$ disjoint sets is the sum of their
sizes.  Then we may write
$$|\cup_{i=1}^k S_i|=|\left(\cup_{i=1}^{k-1}S_i\right)\cup
S_k|=\left(\sum_{i=1}^{k-1}|S_i|\right) +|S_k|=\sum_{i=1}^k|S_i|.$$
Thus whenever the size of the union of $k-1$ disjoint sets is the sum of
their sizes, then the size of a union of $k$ disjoint sets is the sum of
their sizes.  Thus by the principle of mathematical induction, the size
of the union of $n$ disjoint sets is the sum of their sizes for all
$n>1$.  The statement holds trivially when $n=1$ as well.}
\ep

We stated the simplest form of the product principle as 
\begin{quote}If we have a partition of a set $S$ into $m$ blocks, each of
size
$n$, then
$S$ has size $mn$.\end{quote}
In Problem \ref{generalproductprinciple} we gave  a more
general form of the product principle which can be stated as
\index{product principle!general}\index{general product
principle}\begin{quote} Let $S$ be a set of functions $f$ 
from
$[n]$ to some set $X$.  Suppose that
\begin{itemize}
\item there are $k_1$ choices for $f(1)$, and 
\item  suppose that  
for each choice of $f(1)$, $f(2)$, \ldots $f(i-1)$, there are $k_i$
choices for
$f(i)$.
\end{itemize} Then the number of functions in the set
$S$ is $k_1k_2\cdots k_n$.\end{quote}
\bp
\items Prove the general form of the product
principle from the simplest form of the product
principle.\label{generalproductprincipleproof}
\solution{We prove by induction that if $S$ is a set of functions defined on
$[m]$ such that 
\begin{itemize}
\item There are $k_1$ choices for $f(1)$ and
\item when $2\le i\le m$, for each choice of $f(1)$, $f(2)$, \ldots $f(i-1)$,
there are
$k_i$ choices for
$f(i)$,
\end{itemize}
then there are $\prod_{i=1}^m k_i$ functions in $S$.  When $m=1$, the product
is $k_1$ and there are $k_1$ functions in $S$.  Now assume inductively that
when 
 $S'$ is a set of functions defined on $[m-1]$ such that 
\begin{itemize}
\item There are $k_1$ choices for $f(1)$ and
\item when $2\le i\le m-1$, for each choice of $f(1)$, $f(2)$, \ldots
$f(i-1)$, there are
$k_i$ choices for
$f(i)$,
\end{itemize}
then there are $\prod_{i=1}^{m-1} k_i$ functions in $S'$.  Now partition $S$
into $k_1$ sets $S_j$, where $S_j$ is  the set of functions $f$ in $S$ with
$f(n) = y_j$ for each of the $k_n$ values $y_j$ that are possible for
$f(1)$.   Thus $S$ is a union of $k_n$ sets $S_j$ each of
size $\prod_{i=1}^{m-1} k_i$ (by the inductive hypothesis), and so by the
product principle for unions of sets, $S$ has size $\prod_{i=1}^{m} k_i$. 
Therefore, by the principle of mathematical induction, we have proved the
general product principle. }
\ep       
\subsection{Double Induction and Ramsey Numbers}
In Section \ref{Ramseysection} we gave two different descriptions of the
Ramsey number $R(m,n)$.  However if you look carefully, you will see that
we never showed that Ramsey numbers actually exist; we merely described
what they were and showed that $R(3,3)$ and $R(3,4)$ exist by computing
them directly.  As long as we can show that there is some number $R$ such
that when there are $R$ people together, there are either $m$ mutual
acquaintances or $n$ mutual strangers, this shows that the Ramsey Number
$R(m,n)$ exists, because it is the smallest such $R$.  Mathematical
induction allows us to show that one such $R$ is $m+n-2\choose m-1$.  The
question is, what should we induct on, $m$ or $n$?  In other words, do we
use the fact that with $m+n-3\choose m-2 $ people in a room there are at
least $m-1$ mutual acquaintances or $n$ mutual strangers, or do we use
the fact that with at least $m+n-3\choose n-2$ people in a room there are
at least $m$ mutual acquaintances or at least $n-1$ mutual strangers?  It
turns out that we use both.  Thus we want to be able to simultaneously
induct on $m$ and $n$.  One way to do that is to use yet another
variation on the principle of mathematical induction, the {\em Principle
of Double Mathematical Induction}\index{double
induction}\index{induction!double}\index{mathematical induction!double}.
This principle (which can be derived from one of our earlier ones) states
that
\begin{quote}In order to prove a statement about  integers $m$ and $n$, if
we can
\begin{enumerate}
\item Prove the statement when $m=a$ and $n=b$, for  fixed integers $a$
and
$b$
\item Prove the statement when $m=a$ and $n>b$ and when $m>a$  and $n=b$
(for the same fixed integers $a$ and $b$), 
\item Show that the truth of the statement for $m=j$ and $n=k-1$ (with
$j\ge a$ and $k>j$) and the truth of the statement for $m=j-1$ and $n=k$
(with $j>a$ and $k\ge b$) imply the truth of the statement for $m=j$ and
$n=k$,
\end{enumerate}
then we can conclude the statement is true for all pairs of integers $m\ge
a$ and
$n\ge b$.\end{quote}

\bp
\itemesi Prove that $R(m,n)$ exists by proving that if there are
$m+n-2\choose m-1$ people in a room, then there are either at least $m$
mutual acquaintances or at least $n$ mutual strangers.\label{Ramseybound}
\solution{We use double induction on $m$ and $n$ to prove this for
$m,n\ge 2$.    Note first that $R(m,2)=m={m+2-2\choose m-1}$ and $R(2,n)
=n={2+n-2\choose 1}$.  (In words, if there are
$m$ people in a room, then either all $m$ people know each other or there
are two mutual nonacquaintances, and if there are $n$ people in a room,
then either there are two people who know each other or they are all
mutual strangers.)\footnote{Note that we have covered both steps 1 and 2
of a double induction proof now.}  Now assume that whenever there are
$m+n-3\choose m-1$ people in a room there are either at least $m$ mutual
acquaintances or $n-1$ mutual strangers, and that whenever there are at
least $m+n-3\choose m-2$ people in a room there are either at least
$m-1$ mutual acquaintances or $n$ mutual strangers.  Suppose that we have
$m+n-2\choose m-1$ people in a room.  Choose a person, say $P$.  Then
since ${m+n-2\choose m-1} = {m-n-3\choose m-1} +{m+n-3\choose m-2}$,
person $P$ is, by the generalized pigeonhole principle, either acquainted
with $m+n-3\choose m-2$ people or unacquainted with $m+n-3\choose m-1$
people.  In the first case, among the people with whom $P$ is
acquainted, either $m-1$ are mutual acquaintances or $n$ are mutual
strangers.  If $m$ are mutual strangers, we are done, while if $m-1$ are
mutual acquaintances, these $m-1$ people, together with person $P$, are
$m$ mutual acquaintances, in which case we are done as well.  In the
second case, among the $m+n-3\choose m-1$ people with whom person $P$ is
unacquainted, there are either $m$ mutual acquaintances, in which case we
are done, or there are $n-1$ mutual strangers. In this event, these $n-1$
mutual strangers, along with person $P$ make up $n$ mutual strangers. 
Thus in every case, if we know that with $m+n-3\choose m-2$ people in a
room there are either $m-1$ mutual acquaintances or $n$ mutual strangers,
and we know that with $m+n-3\choose m-1$ people in a  room there are
either $m$ mutual acquaintances or $n-1$ mutual strangers, we can
conclude that with $m+n-2\choose m-1$ people in a room there are either
$m$ mutual acquaintances or $n$ mutual strangers.  Therefore by the
principle of double mathematical induction we know that for all $m$ and
$n$ greater than or equal to 2, if there are $m+n-2\choose m-1$ people
in a room, then there are either $m$ mutual acquaintances of $n$ mutual
strangers.  This shows that $R(m,n)$ exists and is no more than
$m+n-2\choose m-1$.}

\itemes Prove that $R(m,n)\le R(m-1,n) + R(m,n-1)$.\label{Ramseyrecurrence}
\solution{If there are $R(m-1,n) +R(m,n-1)$ people in a room, choose one
person, say person $P$.  By the generalized pigeonhole principle, there
are either $R(m-1,n)$ people with whom $P$ is acquainted or $R(m,n-1)$
people with whom person $P$ is unacquainted.  In the first case, among
the people with whom person $P$ is acquainted, there are either $n$
mutual strangers, in which case we are done, or there are  $m-1$
people with whom person $P$ is acquainted.  These $m-1$ people and person
$P$ form $m$ people who are mutually acquainted, and so we have $m$
mutual acquaintances.  On the other hand, if $P$ is unacquainted with
$R(m,n-1)$ people, then among these people, there are either $m$ mutually
acquainted people, in which case we are done, or among these people there
are $m-1$ mutually unacquainted people, and these $m-1$ people together
with $P$ make $m$ mutual strangers.  Thus in every case, if there are
$R(m-1,n)+R(m,n-1)$ people in a room, there are either at least $m$ mutual
acquaintances or at least $n$ mutual strangers.  Therefore $R(m,n)\le
R(m-1,n)+R(m,n-1)$.}

\itemesi \begin{enumerate}
\item What does the equation in Problem
\ref{Ramseyrecurrence} tell us about $R(4,4)$? 
\solution{$R(4,4)\le R(3,4) + R(4,3) =9+9 = 18$.}
\itemitemh Consider 17 people arranged in a circle such that each person is
acquainted with the first, second, fourth, and eighth person to the right
and the first, second, fourth, and eighth person to the left.  can you
find a set of four mutual acquaintances?  Can you find a set of four
mutual strangers?
\solution{You cannot find either.  If there were a set of four mutual
acquaintances, you could assume by symmetry that it includes person 1,
and two people from among those one, two, four, and eight places to the
right.  Thus you can assume your set of four acquaintances contains
person 1 and two from among persons 2, 3, 5, and 9.  However persons 2
and 5, 2 and 9 and 3 and 9 are not acquainted.  Thus three of the
mutually acquainted people are either persons 1, 2, and 3, persons 1, 5
and 9 or persons 1, 3, and 5.  However person 5 is not acquainted with the
person one, two,  or eight places to the left of person 1, so if person 5 is in
the set of mutual acquaintances, then person 14 must be as well. However
person 3 and person 9 are not acquainted with person 14. Thus our set
must contain persons 1, 2, and 3.  However person 3 is not acquainted
with the person one, two,  four, or eight persons to the left of person 1, so
there is no set of four mutual acquaintances.  A similar argument holds for
nonacquaintances.}
\item What is $R(4,4)$?
\solution{18.}\end{enumerate}

\item (Optional) Prove the inequality of Problem
\ref{Ramseybound} by induction on $m+n$. 
\solution{
We want to prove that if $m\ge 2$ and $n\ge 2$, then when there are
$m+n-2\choose m-1$ people in a room, there are either $m$ mutual
acquaintances or $n$ mutual strangers.  If $m+n=4$, then $m=2$ and
$n=2$, and if there are ${2+2-2\choose1}=2$ people in a room, there are
either two  who know each other or two who don't.

Now assume that when $m+n=k-1$, it is the case that if there are $m+n-2\choose
m-1$ people in a room, then there are either $m$ mutual acquaintances or $n$
mutual strangers.  Suppose that $m'+n'=k$ and there are  $m'+n'-2\choose m'-1$
people in a room.   If $n'=2$, then we know that with
${m'+2-2\choose m'-1}=m'$ people in a room, there are either $m'$ mutual
acquaintances or two mutual strangers, and similarly if $m'=2$ there are
either two mutual acquaintances or $n$ mutual strangers among
$m'+n'-2\choose m-1$ people. Thus we may assume that both
$m'$ and $n'$ are greater than two.  Since ${m'+n'-2\choose m-1}=
{(m'-1)+n-2\choose m'-2} +{m+(n-1) -2\choose m-1}$, if there are
$m'+n'-2\choose m'-1$ people in a room, then a given person, say person
$P$, is either acquainted with $(m'-1)+n'-2\choose m'-1$ of them (call
this case 1) or is a stranger with $m'+(n-1)-2\choose m-1$ of them (call
this case 2).  Notice that
$m'-1+n'=k-1$ and $m'+n'-1=k-1$.  Thus in case 1, our inductive hypothesis
tells us that either $m'-1$ of person $P$'s acquaintances are mutually
acquainted, in which case they and person $P$ form $m'$ mutual
acquaintances, or $n'$ of $P$'s acquaintances are mutual strangers, in
which case we have $n'$ mutual strangers.  Similarly in case 2 we have
either $m'$ mutual acquaintances or $n'$ mutual strangers.  Thus by the
principle of mathematical induction, for all values of $m+n$ greater than
or equal to 4, if we have $m+n-2\choose m-1$ people in a room, then we
have either $m$ mutual acquaintances or $n$ mutual strangers, so that
$R(m,n)$ exists and is no more than $m+n-2\choose m-1$.}

\item Use Stirling's approximation (Problem \ref{Stirling'sapproximation})
to convert the upper bound for $R(n,n)$ that you get from Problem
\ref{Ramseybound} to a multiple of a power of an
integer.\label{Ramseybound2}
\solution{$R(n,n)\le {n+n-2\choose m-1}={(2n-2)!\over (n-1)!^2}$.  For
$n$ sufficiently large, this is approximately
\begin{eqnarray*}&&\sqrt{2\pi (2n-2)}(2n-2)^{2n-2}/e^{2n-2}\over
\sqrt{2 \pi (n-1)}(n-1)^{n-1}\sqrt{2 \pi
(n-1)}(n-1)^{n-1}/e^{n-1}e^{n-1}\\
&=&{2^{2n-2}(n-1)^{2n-2}\over\sqrt{\pi
(n-1)}(n-1)^{2n-2}}\\
&=&{1\over \sqrt{\pi
(n-1)}}2^{2n-2}\end{eqnarray*}
}
\ep

\subsection{A bit of asymptotic combinatorics}
Problem \ref{Ramseybound2} gives us an upper bound on $R(n,n)$.  A very
clever technique due to Paul Erd\"os, called the ``probabilistic method,''
will give a lower bound.  Since both bounds are exponential in $n$, they
show that $R(n,n)$ grows exponentially as $n$ gets large.  An analysis of
what happens to a function of $n$ as $n$ gets large is usually called an
{\em asymptotic analysis}.\index{asymptotic combinatorics}  The {\em
probabilistic method}\index{probabilistic method}\index{method!probabilistic},
at least in its simpler forms, can be expressed in terms of averages, so one
does not need to know the language of probability in order to understand it. 
We will apply it to Ramsey numbers in the next problem.  Combined with the
result of Problem
\ref{Ramseybound2}, this problem will give us that
$\sqrt{2}^{\>n}<R(n,n)<2^{2n-2}$, so that we know that the Ramsey number
$R(n,n)$ grows exponentially with~$n$.
\bp 
\itemi Suppose we have two numbers $n$ and $m$.  We consider
all possible ways to color the edges of the complete graph $K_m$
with two colors, say red and blue.  For each coloring, we look at each
$n$-element subset
$N$  of the vertex set $M$ of $K_m$.  Then $N$ together with the edges of
of $K_m$ connecting vertices in $N$ forms a complete graph on $n$
vertices.  This graph, which we denote by $K_N$, has its edges colored by
the original coloring of the edges of $K_m$. 
\begin{enumerate}
\item Why is it that if there is no subset $N\subseteq M$
so that all the edges of $K_N$ are colored the same color, then
$R(n,n)>m$? 
\solution{Another way to say there is no such subset is to say that it
is not possible to find a $K_n$ all of whose edges are red or a $K_n$
all of whose edges are blue.  This means that $R(n,n)>n$.} 
\item To apply the probabilistic method, we are going to compute
the average, over all colorings of $K_m$, of the number of sets
$N\subseteq M$ with
$|N|=n$ such that $K_N$ {\em does} have all its edges the same color.
Explain why it is that if the average is less than 1, then for some
coloring there is no set $N$ such that $K_N$ has all its edges colored
the same color.  Why does this mean that $R(n,n)>m$?
\solution{If the average of $n$ nonnegative integers is less than one,
they cannot all be one or more, so one has to be zero.  Thus in this
context there must be some coloring that has no set $N$ so that $K_N$ has
all its edges colored the same color.}
\item We call a $K_N$ {\em monochromatic}\index{monochromatic subgraph}
for a coloring $c$ of $K_m$ if the color $c(e)$ assigned to edge $e$ is
the same for every edge $e$ of $K_N$.  Let us define ${\rm mono}(c,N)$ to
be 1 if $N$ is monochromatic for $c$ and to be 0 otherwise.  Find a
formula for the average number of monochromatic $K_N$s over all colorings
of $K_m$ that involves a double sum first over all edge colorings $c$ of
$K_m$ and then over all $n$-element subsets $N\subseteq M$ of ${\rm
mono}(c,N)$.
\solution{$${1\over 2^{^{m\choose 2}}}\sum_{c:c\mbox{~is a coloring
of~}K_m}\sum_{N:N\subseteq M,~|N|=n}{\rm mono}(c,N).$$}
\item  Show that your formula for the average reduces to $2{m\choose
n}\cdot2^{-{n\choose 2}}$
\solution{
\begin{eqnarray*}
&&{1\over 2^{^{m\choose 2}}}\sum_{c:c\mbox{~is a coloring
of~}K_m}\sum_{N:N\subseteq M,~|N|=n}{\rm mono}(c,N)\\
&=&{1\over 2^{^{m\choose 2}}} \sum_{N:N\subseteq
M,~|N|=n}\sum_{c:c\mbox{~is a coloring of~}K_m}{\rm mono}(c,N)\\&=&
2^{-{m\choose2}}\sum_{N:N\subseteq
M,~|N|=n}2\cdot2^{{m\choose2}-{n\choose2}}\\
&=& 2{m\choose n}2^{-{n\choose 2}}
\end{eqnarray*}}
\item Explain why $R(n,n)>m$ if ${m\choose n}\le 2^{{n\choose 2} -1}$.
\solution{$R(n,n)>m$ if the average above is less than 1.  Thus $R(n,n)>m$
if
$2{m\choose 2}2^{-{n\choose 2}}<1$, which is equivalent to
${m\choose n}<2^{{n\choose 2}-1}$.}
\itemitemh Explain why $R(n,n)>\root n \of {n!2^{{n\choose 2}-1}}$.
\solution{${m\choose n} <2^{{n\choose 2}-1}$  is the same as
${m^{\underline{n}}\over n!}<2^{{n\choose 2}-1}$.  And since
$m^{\underline{n}}<m^n$, the inequality ${m^{\underline{n}}\over
n!}< 2^{{n\choose 2}-1}$ holds if the inequality ${m^n\over
n!}\le2^{{n\choose2}-1}$ holds. And this last inequality holds if
$m\le\root n
\of {n!2^{{n\choose2}-1}}$ holds.  Thus $R(n,n)>m$ for any $m$ such that
$m\le\root n
\of {n!2^{{n\choose2}-1}}$, which implies that $R(n,n)> \root n
\of {n!2^{{n\choose2}-1}}$.}
\item By using Stirling's formula, show that if $n$ is large enough, then
$R(n,n) > \sqrt{2^n} = \sqrt{2}^{\>n}$
\solution{
Using Stirling's approximation to $n!$ we get $$R(n,n)>\root n \of
{{n^n\over e^n}\sqrt{2\pi n}2^{^{n^2-n-2\over 2}}}={n\over
e}2^{^{n^2-n-2\over 2n}}\root 2n \of{2\pi
n}>2^n/2=\sqrt{2}^{\>n}.$$}
\end{enumerate}
 

\ep


\section{Recurrence Relations}  We have seen in 
Problem \ref{SubsetsByInduction} (or Problem 
\ref{subsetsbysmallestcounterexample} in the Appendix on Induction) that
the number of subsets of an $n$-element set is twice the number of
subsets of an $(n-1)$-element set.  
\bp \item Explain why it is that the number of
bijections from an $n$-element set to an $n$-element set is equal to $n$
times the number of bijections from an $(n-1)$-element subset to an
$(n-1)$-element set.  What does this have to do with Problem
\ref{permutationasbijection}?
\solution{To specify a bijection $f$ from an $n$-element set $\{a_1,a_2, 
\ldots a_n\}$ to an $n$-element set, we have $n$ choices for $f(a_1)$,
and then $b_{n-1}$ choices for how to define $f$ from the elements
$\{a_2,a_3, \ldots,a_n\}$ to the remaining $n-1$ elements of our range. 
By the product principle this gives us $b_n=nb_{n-1}$ for the number
$b_n$ of bijections from an $n$-element set to an $n$-element set.}
\ep
We can summarize these observations as follows.  If $s_n$ stands for the
number of subsets of an $n$-element set, then 
\begin{equation} s_n =2s_{n-1},\label{subsetequation}\end{equation} and if
$b_n$ stands for the number of bijections from an $n$-element set to an
$n$-element set, then \begin{equation} b_n =
nb_{n-1}.\label{bijectionequation}\end{equation}
Equations \ref{subsetequation} and \ref{bijectionequation} are examples
of {\em recurrence equations} or {\em recurrence relations}.  A {\bf
recurrence relation}\index{relation!recurrence}\index{recurrence
relation} or simply a {\bf recurrence}\index{recurrence} is an equation
that expresses the $n$th term of a sequence $a_n$ in terms of values of
$a_i$ for $i<n$.  Thus Equations \ref{subsetequation} and
\ref{bijectionequation} are examples of recurrences. 

\subsection{Examples of recurrence relations}
 Other examples of
recurrences are 
\begin{equation} a_n = a_{n-1} +
7,\label{arithmeticexample}\end{equation}
\begin{equation} a_n =3a_{n-1} + 2^n,\label{geometricdriven}\end{equation}
\begin{equation} a_n = a_{n-1} + 3a_{n-2},\mbox{
and}\label{secondorderlinear}\end{equation}
\begin{equation} a_n= a_1a_{n-1} + a_2a_{n-2}+\cdots +
a_{n-1}a_1.\label{Catalanrecurrence}\end{equation}  A {\bf
solution}\index{recurrence!solution to} to a recurrence relation is a
sequence that satisfies the recurrence relation.  Thus a solution to
Recurrence \ref{subsetequation} is $s_n =2^n$.  Note that
$s_n=17\cdot2^n$ and
$s_n=-13\cdot2^n$ are also solutions to Recurrence \ref{subsetequation}. 
What this shows is that a recurrence can have infinitely many solutions. 
In a given problem, there is generally one solution that is of interest
to us.  For example, if we are interested in the number of subsets of a
set, then the solution to Recurrence \ref{subsetequation} that we care
about is $s_n=2^n$.  Notice this is the only solution we have mentioned
that satisfies
$s_0=1$.
\bp 
\item Show that there is only one solution to Recurrence
\ref{subsetequation} that satisfies $s_0=1$. 
\solution{We prove by induction on $n$ that there is one and only one
value
$s_n$ that satisfies both $s_n=2s_{n-1}$ for $n>0$ and $s_0=1$.  First,
there is clearly one and only one value $s_0$ that satisfies $s_0=1$.  Now
assume that $k>0$ and there is one and only one value $s_{k-1}$ that satisfies
the two equations.  Then $s_k=2s_{k-1}$ is the one and only one value
that satisfies the two equations.  Therefore by the principle of
mathematical induction, for all nonnegative integers $n$ there is one and
only one value
$s_n$ that satisfies the equations $s_0=1$ and $s_k=2s_{k-1}$ for all
$k>0$.  (Note that since we were making a statement about $s_n$ for all
nonnegative integers $n$ it was not appropriate to use $n$ as the dummy
variable in the recursive equation $s_k=2s_{k-1}$.)}

\item A first-order recurrence relation is one which expresses $a_n$ in
terms of $a_{n-1}$ and other functions of $n$, but which does not
include any of the terms $a_i$ for $i<n-1$ in the equation. 
\begin{enumerate}\item Which of
the recurrences \ref{subsetequation} through \ref{Catalanrecurrence} are
first order recurrences?
\solution{ The recurrences \ref{subsetequation}, \ref{bijectionequation},
\ref{arithmeticexample}, and \ref{geometricdriven} are all examples
of first order recurrences.  The recurrences \ref{secondorderlinear} and
\ref{Catalanrecurrence} are not.}
\item Show that there is one and only one sequence $a_n$ that is defined
for every nonnegative integer
$n$, satisfies a given first order recurrence, and satisfies $a_0=a$ for some
fixed constant $a$.
\solution{A first order recurrence will give $a_n$ in terms of $a_{n-1}$, that
is, there will be a function $f$ such that $a_n=f(a_{n-1})$ for all $n>0$.  We
prove by induction that there is one and only one solution to a first order
recurrence that satisfies $a_0=a$ for some fixed constant $a$.  First, there
is one and only one value for $a_0$.  Now suppose that when $n=k-1$, there is
one and only one value possible for $a_{n-1}$.  Then $a_k$ is uniquely
determined by $a_k=f(a_{k-1}$.  Thus the truth of the statement that $a_{k-1}$
is uniquely determined by the equations $a_0=a$ and $a_n=f(a_{n-1})$ implies
the truth of the statement that $a_k$ is determined uniquely by the equations
$a_0=a$ and $a_n=f(a_{n-1})$.  Therefore by the principle of mathematical
induction, $a_k$ is uniquely determined by the equations $a_0=a$ and
$a_n=f(a_{n-1})$ for all nonnegative integers $k$.}
\end{enumerate}
\begin{figure}[htb]\caption{The
Towers of Hanoi Puzzle}\label{Hanoi}\vglue-1in
\begin{center}\mbox{\psfig{figure=Hanoi.eps%,height=1.0in
}}
\end{center}
\end{figure}

\itemi The ``Towers of Hanoi'' puzzle has three rods rising from a
rectangular base with $n$ rings of different sizes stacked in decreasing
order of size on one rod.  A legal move consists of moving a ring from
one rod to another so that it does not land on top of a smaller ring.  If
$m_n$ is the number of moves required to move all the rings from the
initial rod to another rod that you choose, give a recurrence for $m_n$. 
(Hint: suppose you already knew the number of moves needed to solve the
puzzle with $n-1$ rings.)\label{HanoiProblem}
\solution{We can solve the puzzle in one step if there is one ring, so
$m_1=1$.  If $n>0$ and we want to move all the rings from the initial rod to
rod 3, then first we solve the problem of moving all but the bottom ring to
rod 2; this takes $m_{n-1}$ steps, then we move the bottom ring to rod 3, then
we solve the problem of moving all the remaining rings from rod 2 to rod 3. 
Thus we have $m_n=2m_{n-1}+1$. }

\itemi We draw $n$ mutually intersecting circles in the plane so that
each one crosses each other one exactly twice and no three intersect in
the same point.  (As examples, think of Venn diagrams with two or three
mutually intersecting sets.)  Find a recurrence for the number
$r_n$ of regions into which the plane is divided by $n$ circles.  (One
circle divides the plane into two regions, the inside and the outside.) 
Find the number of regions with $n$ circles.  For what values of $n$ can
you draw a Venn diagram showing all the possible intersections of $n$
sets using circles to represent each of the sets? \label{circlesinplane} 
\solution{One circle defines two regions, the inside and outside.  When we
draw a second circle that intersects the first, we can start at one of the
intersection points and go inside the first circle, cutting its region into
two pieces, and then when we leave it we cut the outside region into two
pieces.  This suggests the general pattern.  If we have drawn $n-1$ circles and
start a new one, each time we enter a circle, we start dividing a region into
two pieces.  Each time we leave a circle, we also start dividing a region into
two pieces.  Thus if we have $r_n$ regions with $n$ circles, to get the number
of regions, we note that in going from $n-1$ circles to $n$ circles, we start
with $r_{n-1}$ regions, and divide $2(n-1)$ of them in half, so we get $2n-2$
new regions.  This gives us $r_n=r_{n-1}+2(n-1)$. Notice that by substitution
of the formula $r_{n-1}=r_{n-2} +2(n-2)$, we get $r_n=r_{n-2}+ 2(n-2)
+2(n-1)$, and would guess that $r_n=r_{n-3}+2(n-3)+2(n-2) +2(n-1)$.  This
leads to the conjecture
$$r_n=r_1  +2\cdot1+2\cdot2+\cdots+2\cdot(n-1)=r_1+2\sum_{i=1}^{n-1}
i=2+n(n-1).$$
We can prove this formula by induction.  When $n=1$ we have $2+1\cdot0$
regions.  Assuming that $n-1$ circles give us $2+(n-1)(n-2)$ regions, for $n$
circles we have $2+(n-1)(n-2) +2(n-1)=2+n(n-1)$ regions.  Thus the
correctness of our formula for $n-1$ circles implies its correctness when we
have $n$ circles, so for all positive integers $n$, we get $2+n(n-1)$ regions
determined by $n$ mutually intersecting circles.
Two circles cannot touch more than twice, and if we let some of our $n$ circles
touch just once, or not at all, that would reduce the number of regions we
would get.  Similarly, allowing a circle to go through the intersection point
of two other circles could only reduce the number of regions.  So with $n$
circles we could never have more than $2+n(n-1)$ regions.  In particular
with 4 circles we get just 14 regions, rather than the 16 that would be
required in a Venn diagram for four sets.  We could prove, again by
induction, that $2+n(n-1)<2^n$ for all $n>3$, so it is not possible to draw a
Venn diagram using circles to illustrate the intersections of four or more
sets. 
 }

\ep

\subsection{Arithmetic Series (optional)}
\bp
\item A child puts away two dollars from her allowance each week.  If she
starts with twenty dollars, give a recurrence for the amount $a_n$ of
money she has after $n$ weeks and find out how much money she has at the
end of 
$n$ weeks. \label{childsaving}
\solution{$a_n=a_{n-1} +2$.  Then by substitution $a_n=a_{n-2}+2+2$, and so we
conjecture that $a_n = 20 +2n$.  Since she adds two dollars to her savings
each week for $n$ weeks, she has adds 2n dollars to her original 20, which
proves the formula.  We could have used induction to prove it as well.}

\item A sequence that satisfies a recurrence of the form $a_n=a_{n-1} +c$
is called an {\em arithmetic progression}\index{arithmetic
progression}\index{progression!arithmetic}. Find a formula in terms of
the initial value
$a_0$ and the common difference $c$ for the term
$a_n$ in an arithmetic progression and prove you are
right.\label{arithmeticprogression}
\solution{$a_n =a_o+cn$.  The formula is valid with $n=0$, and if $a_{n-1}=a_0
+c(n-1)$, then $a_n = a_0 +c(n-1) +c =a_0+cn$.  Therefore the fact that
$a_{n-1}=a_0+ca_{n-1}$ implies the fact that $a_n=a_0+cn$.  Therefore by the
principle of mathematical induction, $a_n=a_0+cn$ for all nonnegative integers
$n$.}
\item A person who is earning \$50,000 per year gets a raise of \$3000 a
year for $n$ years in a row.  Find a recurrence for the amount $a_n$ of
money the person earns over $n+1$ years. What is the total amount of money
that the person earns over a period of $n+1$ years?  (In
$n+1$ years, there are
$n$ raises.) \label{constantraise}
\solution{By Problem \ref{arithmeticprogression} we saw that if $b_n$ is the
salary in year $n$, then $b_n=50,000 + 3000n$. If $a_n$ is the total amount
earned over the period of from year 0 through the end of year $n$, a period of
$n+1$ years,  then $a_n=a_{n-1}+b_n=a_{n-1}+ 50,000+3000n$.  Further,
$a_n=\sum_{i=0}^n b_i=\sum_{i=0}^n50,000 +3000i = 50,000(n+1)+3000\sum_{i=0}^n
i= 50,000(n+1) + 1500(n(n+1)$.}
\item An {\em arithmetic
series}\index{series!arithmetic}\index{arithmetic series} is a sequence
$s_n$ equal to the sum of the terms
$a_0$ through
$a_n$ of of an arithmetic progression.  Find a recurrence for the sum
$s_n$ of an arithmetic progression with initial value $a_0$ and common
difference
$c$ (using the language of Problem \ref{arithmeticprogression}).  Find a
formula for general term $s_n$ of an arithmetic  series.
\solution{$s_n=\sum_{i=0}^n a_0 +ci =(n+1)a_0+c\sum_{i=0}^n i = (n+1)a_0
+cn(n+1)/2$.}
\ep


\subsection{First order linear recurrences}  Recurrences such as those in
Equations \ref{subsetequation} through \ref{secondorderlinear} are called
{\em linear recurrences}, as are the recurrences of Problems
\ref{HanoiProblem} and
\ref{circlesinplane}.  A {\bf linear recurrence}\index{recurrence!linear}\index{linear
recurrence} is one in which $a_n$ is expressed as a sum of functions of
$n$ times values of (some of the terms) $a_i$ for $i<n$ plus (perhaps)
another function (called the {\em driving function}\index{driving
function}\index{function!driving}) of
$n$.  A linear equation is called {\em homogeneous}\index{homogeneous
linear recurrence}\index{linear
recurrence!homogeneous}\index{recurrence!linear homogeneous} if  the
driving function is zero (or, in other words, there is no driving
function).  It is called a {\em constant coefficient linear
recurrence}\index{linear recurrence!constant coefficient}\index{constant
coefficient linear recurrence} if the functions that are multiplied by the
$a_i$ terms are all constants (but the driving function need not
be constant).
\bp 
\item Classify the recurrences in Equations \ref{subsetequation} through
\ref{secondorderlinear} and Problems \ref{HanoiProblem} and
\ref{circlesinplane} according to whether or not they are constant
coefficient, and whether or not they are
homogeneous.\label{classifyrecurrences}
\solution{Recurrence \ref{subsetequation} is first order, linear, constant
coefficient, and homogeneous.  Recurrence \ref{bijectionequation} is first
order, linear, and homogeneous, but not constant coefficient.
 Recurrence \ref{arithmeticexample} is first order, linear, constant
coefficient but not homogeneous.
Recurrence \ref{geometricdriven} is first order, linear, constant coefficient
but not homogeneous.  Recurrence \ref{secondorderlinear} is not first order
(it is second order), is linear, constant coefficient and homogeneous. 
The recurrence of Problem \ref{HanoiProblem} is first order, linear, and
constant coefficient, and that of Problem \ref{circlesinplane} is first
order, linear, and constant coefficient.}

\iteme As you can see from Problem \ref{classifyrecurrences} some
interesting sequences satisfy first order linear recurrences, including
many that have constant coefficients, have constant driving term, or are
homogeneous. Find a formula in terms of $b$, $d$, $a_0$ and $n$ for the
general term $a_n$ of a sequence that satisfies a constant coefficient first
order linear recurrence $a_n = ba_{n-1} + d$ and prove you are
correct.  If your formula involves a summation, try to replace the
summation by a more compact expression.  \label{firstordlinconst}
\solution{Note that by the formula, $a_{n-1}=ba_{n-2} +d$.  Substituting this
into the original equation for $a_n$ gives $a_n=b^2a_{n-2}+bd +d$.  Repeating
this kind of substitution gives us $a_n=b^3a_{n-3} +b^2d +bd +d$.  This
suggests that $a_n = a_0b^n +\sum_{i=0}^{n-1}db^i$.  We would guess the same
formula by writing out the first few values of $a_i$, namely, $a_0$, $a_0b+d$,
$a_0b^2+db+d$, $a_0b^3+b^2d+bd+d$, and so on.  We prove our general formula by
induction on $n$.  It is clearly true when $n=0$ as there are no terms in the
sum and $b^0=1$.  If we assume the formula is true when $n=k-1$, we may write
\begin{eqnarray*}
a_k &=& ba_{k-1}+d\\
&=&b(a_0b^{k-1}+ b\left(\sum_{i=0}^{n-2}db^i\right) +d\\
&=&a_0b^n +\sum_{i=0}^{n-1}db^i
\end{eqnarray*}
Thus the truth of our formula for $n=k-1$ implies its truth for $n=k$ and
therefore by the principle of mathematical induction, it is true for all
nonnegative integers $n$.  

We can give a more compact expression for the sum
$\sum_{i=0}^{n-1}db^i=d\sum_{i=0}^{n-1}b^i$.  Recall from algebra that
$(1+x)(1-x)= 1-x^2$, $(1+x+x^2)(1-x) = 1-x^3$, and in general $(1+x+x^2+\cdots
x^{n-1}))(1-x) = 1-x^n$.  If you do not recall this formula, you can prove it
by induction, or observe that 
\begin{eqnarray*}
&&(1+x+x^2+\cdots
x^{n-1}))(1-x)\\ &=& (1+x+x^2+\cdots
x^{n-1})\cdot 1-(1+x+x^2+\cdots
x^{n-1})\cdot x\\&=& 1+x+x^2+\cdots x^{n-1}-(x+x^2+\cdots+x^n)\>=\> 1-x^n.
\end{eqnarray*}
Dividing the first and last terms by $1-x$ gives us
$$\sum_{i=1}^{n-1}x^i={1-x^n\over 1-x}.$$  Using this in our formula for $a_n$
gives us
$a_n =a_ob^n+d{1-b^n\over 1-b}.$ This is valid except in the case $b=1$ (in
our computation with $x$ above, we would be dividing by 0.)  If $b=1$ we get
$a_n =a_0 +nd$ for the sum.}
\ep

\subsection{Geometric Series}
A sequence that satisfies a recurrence of the form $a_n=ba_{n-1}$ is
called a {\em geometric progression}\index{geometric
progression}\index{progression!geometric}.  Thus the sequence satisfying
Equation
\ref{subsetequation}, the recurrence for the number of subsets of an
$n$-element set, is an example of a geometric progression. From your
solution to Problem \ref{firstordlinconst}, a geometric progression has
the form $a_n=a_0b^n$.  In your solution to Problem
\ref{firstordlinconst} you may have had to deal with the sum of a
geometric progression in just slightly different notation, namely
$\sum_{i=0}^{n-1}db^i$.  A sum of this form is called a {\bf (finite)
geometric series}.\index{geometric series}\index{series!geometric}

\bp 
\item Do this problem only if your final answer (so far) to Problem
\ref{firstordlinconst} contained the sum   $\sum_{i=0}^{n-1}db^i$.
\label{sumgeometricseries} 
\begin{enumerate}\item Expand $(1-x)(1+x)$.  Expand $(1-x)(1+x+x^2)$. 
Expand $(1-x)(1+x+x^2+x^3)$.  
\solution{$(1-x)(1+x)=1-x^2$.  $(1-x)(1+x+x^2)=1-x^3$. 
$(1-x)(1+x+x^2+x^3)=1-x^4$.}
\item What do you expect $(1-b)\sum_{i=0}^{n-1} db^i$ to be?  What
formula for $\sum_{i=0}^{n-1}db^i$ does this give you?  Prove that you
are correct.
\solution{We expect $(1-b)\sum_{i=0}^{n-1} db^i$ to be $d(1-b^n)$.  If
$b\not=1$, this gives us  $\sum_{i=0}^{n-1}db^i=d{1-b^n\over 1-b}.$  We can
prove this by induction on $n$.  If $n=0$ we get 0 for $1-b^n\over 1-b$, and
also for the sum $\sum_{i=0}^{-1}db^i$, since that sum has no terms.  Assuming
the formula holds when $n=k-1$, we may write
\begin{eqnarray*}&&\sum_{i=0}^{k-1} db^i\\
&=&\sum_{i=0}^{k-2}db^i+db^{k-1}\\
&=&d{1-b^{k-1}\over 1-b}+db^{k-1}\\
&=&{ d-db^{k-1}  +db^{k-1}-db^k\over 1-b}\>=\> d{1-b^k\over 1-b}.
\end{eqnarray*}
Since the truth of the formula for $n=k-1$ implies its truth for $n=k$, by the
principle of mathematical induction the formula is true for all nonnegative
integers $n$. If
$b=1$ we get the formula
$\sum_{i=0}^{n-1}db^i=dn$.}
\end{enumerate}
\ep 

In Problem \ref{firstordlinconst} and perhaps \ref{sumgeometricseries}
you proved an important theorem.

\begin{theorem} If $b\not=1$ and $a_n=ba_{n-1} +d$, then $\displaystyle a_n =
a_0b^n + d{1-b^n\over 1-b}$. If $b=1$, then,   $\displaystyle a_n =
a_0 +nd$ \end{theorem}

\begin{corollary} If $b\not=1$, then $\displaystyle \sum_{i=0}^{n-1}b^i =
{1-b^n\over 1-b}$.  If $b=1$, $\displaystyle \sum_{i=0}^{n-1}b^i =n$.
\end{corollary}




\section{Graphs and Trees}
\subsection{Undirected graphs} \label{graphsection}
In Section \ref{Ramseysection} we introduced the idea of a directed
graph.  Graphs consist of vertices and edges.  We describe vertices and
edges in much the same way as we describe points and lines in geometry:
we don't really say what vertices and edges are, but we say what they
do.  We just don't have a complicated axiom system the way we do in
geometry.  A {\bf graph}\index{graph} consists of a set $V$ called a
vertex set and a set $E$ called an edge set.  Each member of $V$ is
called a {\em vertex}\index{vertex} and each member of $E$ is called an
{\em edge}.\index{edge} Associated with each edge are two (not necessarily
different) vertices called its endpoints.    We draw pictures of graphs by
drawing points to represent the vertices and line segments (curved if we
choose) whose endpoints are at vertices to represent the edges. In
Figure
\ref{Threegraphs} we show three pictures of graphs.
\begin{figure}[htb]\caption{Three different
graphs}\label{Threegraphs}
\begin{center}\mbox{\psfig{figure=threegraphs.eps%,height=1.0in
}}
\end{center}
\end{figure}
Each grey circle in the figure represents a vertex; each line segment
represents an edge.  You will note that we labelled
the vertices; these labels are names we chose to give the vertices.  We
can choose names or not as we please.  The third graph also shows that it
is possible to have an edge that connects a vertex (like the one labelled
$y$) to itself or it is possible to have two or more edges (like those
between vertices $v$ and $y$) between two vertices.  The {\em
degree}\index{vertex!degree of}\index{degree of a vertex} of a vertex is
the number of times it appears as the endpoint of edges; thus the degree
of
$y$ in the third graph in the figure is four.
\bp 
\itemm In the graph on the left in Figure \ref{Threegraphs}, what is the
degree of each vertex?  
\solution{The degree of vertex 1 is one, of vertex 2 is two, of vertex 3 is
three, of vertex 4 is three, of vertex 5 is one, of vertex 6 is one, of vertex
7 is two, of vertex 8 is one.}

\itemm  For each graph in Figure \ref{Threegraphs} is the number of
vertices of odd degree even or odd?
\solution{In all three cases it is even.}

\itemesi
The sum of the degrees of the vertices of a (finite) graph is related in a
natural way to the number of edges.  
\begin{enumerate}
\item What is the relationship?  \solution{The sum of the degrees of the
vertices is twice the number of edges.}
\item Find a proof that what you say is correct that uses induction on
the number of edges.  Hint:  To make your inductive step, think about
what happens to a graph if you delete an edge. \solution{If a graph has
no edges, then the sum of the degrees of the vertices is 0, which is
twice the number of edges.  Now suppose that whenever a graph has $n-1$
edges, the sum of the degrees of the vertices is twice the number of edges. 
Let
$G$ be  a graph with $n$ edges, and delete an edge from $G$ to get $G'$.  The
the sum of the degrees of $G'$ is $2(n-1)$, and adding the edge back into
$G'$ to get $G$ either increases the degrees of exactly two vertices by
one each or increases the degree of one vertex by 2.  Thus the sum of the
degrees of the vertices of $G$ is $2n$, which is twice the number of
edges.  Thus by the principle of mathematical induction, for all nonnegative
integers
$n$, if a graph has $n$ edges, then the sum of the degrees of the vertices is
twice the number of edges.}
\item Find a proof that what you say is correct that uses induction on
the number of vertices.
\solution{If a graph has no vertices, then the sum of the degrees of the
vertices is 0, which is twice the number of edges.  Now suppose that whenever
a graph has
$n-1$ vertices, the sum of the degrees is twice the number of edges.  Let $G$
be a graph with $n$ vertices.  Delete one vertex $x$ and all edges it is
incident with.  By our inductive hypothesis the sum of the degrees of the
vertices of the resulting graph is twice the number of edges. Now replace the
vertex
$x$ and add in the edges connecting it to other vertices in the graph.  Each
time you add in an edge, you increase the degree of $x$ by one and of one other
vertex by one. Finally add in any edges connecting $x$ to itself.  Each of
these increases the degree of $x$ by two.  Thus for each edge we added, the
sum of the degrees of the vertices increased by two, so the sum of the degrees
of the vertices of
$G$ is still twice the number of edges.  Therefore by the principle of
mathematical induction, for all nonnegative integers $n$, if a graph has $n$
vertices, then the sum of the degrees of its vertices is even.}
\item Find a proof that what you say is correct that does not use
induction.
\solution{The sum of the degrees of the vertices is the sum over all
edges of the number of times that edge touches a vertex, which is twice
the number of edges.}
\end{enumerate}

\itemes What can you say about the number of vertices of odd degree in a
graph?
\solution{The number of vertices of odd degree must be even, because otherwise
the sum of the degrees of the vertices would be odd.}
\ep



\subsection{Walks and paths in graphs}
A {\em walk} in a graph is an alternating sequence $v_0e_1v_1\ldots
e_iv_i$ of vertices and edges such that edge $e_i$ connects vertices
$v_{i-1}$ and $v_i$.  A graph is called connected if, for any pair of
vertices, there is a walk starting at one and ending at the other.
\bp\item
Which of the graphs in Figure \ref{Threegraphs} is
connected?\label{connectedanddisconnected} 
\solution{They first two connected; the third is not.}

\itemm 
A {\em path} in a graph is a walk with no repeated vertices.  Find the
longest path you can in the third graph of Figure \ref{Threegraphs}.
\solution{The path from $y$ to $v$ to $x$ to $w$ is a typical longest path.
There are quite a few others.  Notice you have two choices for the edge to use
to get from $y$ to $v$. } 
 
\itemm A {\em cycle} in a graph is a walk whose first and last vertex are
equal but which has no other repeated vertices.  Which graphs in Figure
\ref{Threegraphs} have cycles?  What is the largest number of edges in a
cycle in the second graph in Figure \ref{Threegraphs}?  What is the
smallest number of edges in a cycle in the third graph in Figure
\ref{Threegraphs}?\solution{The second and third graphs have cycles.  The
largest number of edges in a cycle in the second graph is six; the smallest
number of edges in a cycle in the third graph is one.}

\itemm A connected graph with no cycles is called a {\bf
tree}\index{tree}.  Which graphs, if any, in Figure \ref{Threegraphs} are
trees?
\solution{The first graph is a tree.}
\ep

\subsection{Counting vertices, edges, and paths in trees}
\bp 
\itemesi
Draw some trees and on the basis of your examples, make a conjecture
about the relationship between the number of vertices and edges in a
tree.  Prove your conjecture.  (Hint:  what happens if you choose an edge
and delete it, but not its endpoints?)\label{Noverticesandedgesoftree}
\solution{The number of edges of a tree is one less than the number of
vertices.  We prove this by strong induction on the number of edges.  First, if
a tree has no edges, it can have only one vertex (otherwise it is not
connected).  Thus the number of edges is one less than the number of vertices. 
Now suppose that if a tree has fewer than $n$ edges, it the number of edges is
one less than the number of vertices.  Choose an edge $e$ with endpoints $x$
and $y$ in the tree and remove it.  The resulting graph is not connected, for
if it were, the path remaining between the endpoints of $e$, together with
$e$, would form a cycle.  Every vertex in the tree is connected to $x$ by a
path; those connected by a path not using $e$ still remain connected to $x$. 
Similarly those connected to $y$ by a path in the tree that does not use the
edge $e$ remain connected to $y$ after $e$ is deleted. Every vertex is
connected to $x$ by a path in the tree and every vertex is connected to $y$ by
a path in the tree. If a vertex $v$ were connected to $x$ by a path using the
edge $e$ and to $y$ by a path using the edge $e$, that would mean that we have
a path from
$v$ to $x$ and then through $e$ to $y$ and also a path from $v$ to $y$ and
then through $e$ to $x$.  Thus taking $e$ out of these two paths would give us
a path from $v$ to $x$ and a path from $v$ to $y$, neither of which used
$e$.  Thus when we removed the edge $e$, there would still be a path from $x$
to $y$, and adding $e$ to this path would give a cycle.  Therefore each vertex
is connected to either $x$ or $y$, but not both, by a path that does not use
$e$.  Therefore when we remove $e$, the graph that remains consists of two
trees, the tree of all vertices connected to $x$ and the tree of all vertices
connected to $y$.  Each of these trees has fewer edges than the original tree,
so if they have $m_1$ and $m_2$ vertices respectively, they have, by the
inductive hypothesis, $m_1-1$ and $m_2-1$ edges respectively.  But together
they have all the vertices of the original tree, so the original tree has
$m_1+m_2$ vertices, and has $m_1-1+m_2-1 +1=m)_1+m_2-1$ edges, the edges of
each of the two smaller trees as well as the edge $e$.  Therefore the number
of edges of the original tree is one less than the number of vertices. 
Therefore by the strong principle of mathematical induction, the number of
edges of a tree is one less than the number of vertices.}

\itemes What is the minimum number of vertices of degree one in a finite
tree?  What is it if the number of vertices is bigger than one?  Prove
that you are correct.
\solution{The minimum is zero, which happens with a tree
with one vertex.  If the tree has more than one vertex, the minimum number of
vertices of degree one is two.  To prove this, we prove that every tree with
two or more vertices has at least two vertices of degree two. Note that a
tree with two vertices has exactly two vertices of degree 2.  Now take a tree
with more than two vertices.  Remove an edge
$e$ without removing its endpoints.  As in the solution to Problem
\ref{Noverticesandedgesoftree} this gives two trees.  We may assume
inductively that each has at least two vertices of degree 1, or else is a
single vertex.  When we put $e$ back in, it connects one vertex in one tree to
one in the other.  If both these vertices have degree 1 in their trees, there
will be at least one vertex of degree 1 remaining in each tree, so there will
be at least two vertices of degree 1 in the tree we get.  If exactly one of
these vertices is a tree with one vertex after the removal of  $e$, when we
connect it to the other tree, we will remove at most one increase the degree
of at most one vertex of degree 1 and will create a new vertex of degree 1, so
the tree that results still has at least two vertices of degree 1.  Therefore  
by the strong principle of mathematical induction, every tree with more than
two vertices has at least two vertices of degree 1.  Since a two-vertex tree
has two vertices of degree 1, the minimum number of vertices of degree 1 in a
tree with two or more vertices is two.  (In fact a path with $n$ vertices is a
tree and it has exactly two vertices of degree one also.)}

\itemesi In a tree, given two vertices, how many paths can you find between
them?  Prove that you are correct.
\solution{Exactly one.  Suppose there were two distinct paths $P_1$ and $P_2$
from $x$ to $y$.  As they leave $x$, they might leave on the same edge or on
different edges.  However, since they are different, there must be some first
vertex $x'$ on both paths so that when leave $x'$ (as we go from $x$ to $y$),
they leave on different edges.  Then since they must both enter $y$, there must
be some first vertex $y'$, following $x'$ on both paths as we go from $x$ to
$y$, such that the two paths enter $y'$ on two different edges.  Then the
portion of path 1 from $x'$ to $y'$ followed by the portion of path 2 from
$y'$ to $x'$ will be a cycle.  This is impossible in a tree, so the
supposition that there were two distinct paths is impossible.}

\itemih  How many trees are there on the vertex set $\{1,2\}$?  On the
vertex set $\{1,2,3\}$?  When we label the vertices of our tree, we
consider the tree which has edges between vertices 1 and 2 and between
vertices 2 and 3 different from the tree that has edges between vertices
1 and 3 and between 2 and 3.  See Figure \ref{differenttrees}.
\begin{figure}[htb]\caption{The three labelled
trees on three vertices}\label{differenttrees}
\begin{center}\mbox{\psfig{figure=threetrees%,height=1.0in
}}
\end{center}
\end{figure}  How many
(labelled) trees are there on four vertices?  You don't have a lot of data
to guess from, but try to guess a formula for the number of labelled trees
with vertex set
$\{1,2,\cdots,n\}$.  (If you organize carefully, you can figure out how
many labelled trees there are with vertex set $\{1,2,3,4,5\}$ to help you
make your conjecture.)  Given a tree with two or more vertices, labelled
with positive integers, we define a sequence $b_1,b_2,\ldots$ of
integers inductively as follows:  If the tree has two vertices, the
sequence  consists of one entry, namely the label of the vertex with the
larger label.  Otherwise, let
$a_1$ be the lowest numbered vertex of degree 1 in the tree.  Let $b_1$
be the label of the unique vertex in the tree adjacent to $a_1$ and write
down
$b_1$.  
Given $a_{i-1}$ and $b_{i-1}$, let $a_i$ be the lowest numbered vertex of
degree 1 in the tree you get by deleting $a_1$ through $a_{i-1}$ and let $b_i$
be the unique vertex in this new tree adjacent to $a_i$.
We use $b$ to stand for the sequence of $b_i$s we get in this way. For example
in the tree (the first graph) in Figure
\ref{Threegraphs}, $a_1$ is 1 and the sequence $b$ is 234478. (If you are
unfamiliar with inductive (recursive) definition, you might want to write
down some other labelled trees on eight vertices and construct the
sequence of $b_i$s.) How long will the sequence
$b$ be if it is applied to a tree with
$n$ vertices (labelled with 1 through
$n$)?  What can you say about the last member of the sequence of $b_i$s? 
Can you tell from the sequence of
$b_i$s what
$a_1$ is?  Find a bijection between labelled trees and something you can
``count" that will tell you how many labelled trees there are on $n$
labelled vertices.\label{Prufer}
\solution{There is one labelled tree on two vertices.  We know there are three
labelled trees on three vertices, and they all are paths.  The difference
among them is which vertex is the central vertex on the path.  On four
vertices a tree either has a vertex of degree 3 (there are four such trees) or
it is a path, in which case there are six choices for the two vertices of
degree 2, and for each choice of these two vertices, there are two different
ways to attach the remaining vertices to them as vertices of degree 1.  Thus
there are $12+4=16$ trees on four vertices.  Assuming this is not enough data
for a good guess, note that on five vertices, we either have a vertex of
degree 4 (there are five such trees), or we have a vertex of degree three
which must be adjacent to a vertex of degree two in order to have five
vertices.  There are $5\cdot4=20$ ways to choose these two vertices; then
three more choices we can make for the degree one vertex attached to the degree
2 vertex.  Thus we have 60 trees with a vertex of degree three.  If we have
neither a vertex of degree four nor a vertex of degree three, then the tree is
a path.  We have ${5\choose2}=10$ ways to choose the two vertices of degree
one, and then there are $3!=6$ ways to arrange the remaining vertices along
the path, so we have 60 paths.  Thus we have $125$ trees on five vertices. 
These computations suggest there are $n^{n-2}$ labelled trees on $n$ vertices,
which is what we shall prove with the sequences of $a$\/s and $b$\/s.

On a tree with $n$ vertices, the sequence $b$ will have length $n-1$.  The
last member of the sequence $b$ will be $n$.  Vertex $n$ can not be in the
sequence $a$, because the tree that remains after we delete an $a_i$ will have
at least two vertices of degree 1, so the one of smaller degree will be
$a_{i+1}$.  Thus we never delete the vertex $n$ from the tree.  Therefore when
we choose the last $b$, we have vertex $n$ and one other vertex, so the other
vertex is our $a$-vertex and $n$ is the vertex adjacent to it.  $a_1$ will be
the smallest number that is not in the sequence of $b$'s.  This tells us one
edge of the tree, namely the edge between $a_1$ and $b_1$.  The vertex $a_2$
will be the smallest number different from $a_1$ not in the sequence $b_2$
through
$b_{n-1}$. 
 In general, $a_i$
will be the smallest vertex different from $a_1$ through $a_{i-1}$ not in the
sequence
$b_i$ through
$b_{n-1}$, which gives us all $n-1$ edges of the tree (edge $i$ goes from
$a_i$ to
$b_i$).  Thus there is a bijection between trees and the sequences $b_1$
through $b_{n-1}$. But since $b_{n-1}=n$, there is also a bijection between
trees and the sequences $b_1$ through $b_{n-2}$.  But given a sequence of
numbers $c_1,c_2,\ldots,c_{n-2},c_{n-1}$, all between 1 and $n$ and with
$c_{n-1}=n$, there is always a smallest number $a_1$ not in the sequence, and
given
$a_1,
a_2, \ldots a_{i-1}$, there is always a smallest number 
not in the sequence $c_i$ through $c_{n-1}$ and different from the $a_i$s
already chosen, so we can construct the edges from $a_i$ to $c_i$.  Further,
if we start with the edge from $a_{n-1}$ to $c_{n-1}$ and work backwards, we
will always have a connected graph and will always be adding a vertex of
degree 1 to it, so we will have no cycles.  Therefore we will get  a tree. 
Thus we have a bijection between labelled trees on $n$ vertices and sequences
of length $n-2$ consisting of members of $[n]$.  There are $n^{n-2}$ such
sequences, and thus $n^{n-2}$ labelled trees on $n$ vertices.}

\ep
The sequence $b_1,b_2,\ldots, b_{n-2}$ in Problem \ref{Prufer} is called a
{\em Pr\"ufer coding} or {\em Pr\"ufer code} for the tree.  There is a good bit
of interesting information encoded into the Pr\"ufer code for a tree.

\bp
\item What can you say about the Pr\"ufer code for a tree with exactly
two vertices of degree 1? Does this characterize such trees? 
\solution{It consists of
$n-2$ distinct numbers between 1 and $n$.  Any such Pr\"ufer code is the code
of a tree with exactly two vertices of degree one.}

\itemi What can you determine about the degree of
the vertex labelled $i$ from the Prufer code of the tree?
\solution{The degree of a vertex in a tree is one more than the number of
times the vertex appears in the Pr\"ufer code of the tree.}

\itemi What is the number of (labelled) trees on $n$ vertices with three
vertices of degree 1? (Assume they are labelled with the integers 1
through $n$.)
\solution{There are $n\choose 3$ ways to choose the three vertices of degree
one.  Each of the other $n-3$ vertices must appear in the Pr\"ufer Code, so one
must appear twice.  We have $n-3$ ways to choose that one vertex and
${n-2\choose 2}{n-4\choose1}{n-5\choose1}\cdots{1\choose1} ={(n-2)!\over 2!}$
ways to choose which of the
$n-2$ places to use for which vertices in the Pr\"ufer code. 
Thus there are
${n\choose3}(n-3){(n-2)!\over 2}= {n!(n-2)(n-3)\over 12}$ labelled trees with
three vertices of degree one.}




\ep
\subsection{Spanning trees}

Many of the applications of trees arise from trying to find an efficient
way to connect all the vertices of a graph.  For example, in a telephone
network, at any given time we have a certain number of wires (or
microwave channels, or cellular channels) available for use.  These wires
or channels go from a specific place to a specific place. Thus the wires
or channels may be thought of as edges of a graph and the places the wires
connect may be thought of as vertices of that graph.  A tree whose edges
are some of the edges of a graph $G$ and whose vertices are all of the
vertices of the graph $G$ is called a {\bf spanning tree}\index{spanning
tree}\index{tree!spanning} of $G$.  A spanning tree for a telephone
network will give us a way to route calls between any two vertices in the
network. In Figure \ref{spanningtrees} we show a graph and all its
spanning trees.
\begin{figure}[htb]\caption{A graph
and all its spanning trees.}\label{spanningtrees}
\begin{center}\mbox{\psfig{figure=spanningtrees.eps,height=1.5in
}}\end{center}\end{figure}


\bp
\item Show that every connected graph has a spanning tree.  It is
possible to find a proof that starts with the graph and works ``down"
towards the spanning tree and to find a proof that starts with just the
vertices and works ``up" towards the spanning tree.  Can you find both
kinds of proof?
\solution{Here are three proofs:

Start with a connected graph, and if you can
find a cycle, remove one edge of that cycle.  Repeat this process until you
get a tree.  You will get a tree, because removing an edge of a cycle reduces
the number of cycles but leaves the graph connected.  This tree will be a
spanning tree of the original graph.

Start with the original vertex set and no edges for a graph $H$.  Go through
the edges of the original graph $G$ one at a time, and if you can add an edge
of $G$ to the graph $H$ without creating a cycle, do so.  Otherwise
discard the edge and go on to the next one.  This will give you a graph $H$
with no cycles, and if it were not connected, there would be an edge in $G$
between two vertices not yet connected in $H$.  (If there weren't, the graph
you just constructed would be connected.)  Thus you get a spanning tree.

Start with vertex 1 and no edges as a graph $H$.  Take an edge from it to
another vertex.  Add that edge and vertex to $H$.  Now take an edge from one of
the vertices you currently have to yet another vertex of the original graph.
Add that vertex and edge to $H$. Repeat this process until you can find no such
edge.  You will get a tree because each edge you add connects a vertex of
degree 1 to a tree you have already constructed.  Since the original graph is
connected, there must always be an edge from the current set of vertices you
are considering to something not in that set.}
\ep

\subsection{Minimum cost spanning trees}
Our motivation for talking about spanning trees was the idea of finding a
minimum number of edges we need to connect all the edges of a
communication network together.  In many cases edges of a communication
network come with costs associated with them.  For example, one
cell-phone operator charges another one when a customer of the first uses
an antenna of the other.  Suppose a company has offices in a number of
cities and wants to put together a communication network connecting its
various locations with high-speed computer communications, but to do so at
minimum cost.  Then it wants to take a graph whose vertices are the
cities in which it has offices and whose edges represent possible
communications lines between the cities.  Of course there will not
necessarily be lines between each pair of cities, and the company will
not want to pay for a line connecting city $i$ and city $j$ if it can
already connect them indirectly by using other lines it has chosen.  Thus
it will want to choose a spanning tree of minimum cost among all spanning
trees of the communications graph.   For reasons of this application, if
we have a graph with numbers assigned to its edges, the sum of the numbers
on the edges of a spanning tree of
$G$ will be called the {\em cost}\index{spanning tree!cost
of}\index{tree!spanning!cost of}\index{cost of a spanning tree} of the
spanning tree.  
\bp
\itemi Describe a method (or better, two methods different in at least one
aspect) for finding a spanning tree of minimum cost in a graph whose edges
are labelled with costs, the cost on an edge being the cost for including
that edge in a spanning tree.  Prove that your method(s)
work.\index{minimum cost spanning tree}\index{spanning tree!minimum
cost}\index{tree!spanning!minimum cost}\label{mincostspantree}
\solution{Start with  vertex, number it vertex 1, and choose
the least costly edge leaving it.  Number the new vertex 2.  Given your current
set of vertices and edges, choose the least costly edge leaving that set, and
if your set has $i-1$ vertices, label the new vertex $i$.  You will always
have a tree as you go along since you are adding a vertex of degree 1 to a
tree you already have.  You will get a spanning tree because the graph you
start with is connected.  If your tree did not have the lowest cost among all
spanning trees, there would be some smallest $i$ such that there is an edge
from vertex
$i$ in a least cost spanning tree that is not in your tree.  Since we chose
the least $i$, the edge we just chose from vertex $i$ goes to a
higher-numbered vertex.  Thus you could have chosen that edge as you were
constructing your tree, so there cannot be a tree of lower total cost than the
one you chose.

Alternatively, start with the vertex set of the graph and no edges.  Choose an
edge of least cost.  Repeat the following until it cannot be repeated. Given
the set of edges you have so far, choose an edge of least cost among all edges
that do not form a cycle with edges you already have.  The graph you get will
have no cycles, and it will have to be connected, because otherwise there
would be an edge in the original graph that connects two vertices in the graph
you just constructed.  Thus you will have a tree.  Suppose there is another
tree with lower total cost.  Choose such a tree with as many edges as common
with your tree as possible.  Then there is some edge
$e$ of this tree of lowest cost among all edges connecting two vertices that
are not connected by an edge in your tree.  Suppose the cost of this edge is
$c$.  Since these vertices are connected by some path in your tree, when you
were considering edges of cost
$c$, these two vertices were already connected by a path in your tree.  There
must be some edge $f$ on that path not in the least cost tree.  The edge $f$
was already in your tree while you were considering edges of cost $c$, so its
cost is no more than $c$.  Adding $f$ to the tree of least cost gives a
cycle.  All edges on that cycle that are not in your tree have cost at least
$c$ by our choice of $e$.  But since $f$ was from your tree, there must be
some edge $g$ of the cycle that is  in the least cost tree that but not in
your tree.  Removing $g$ from the least cost tree and adding $f$ cannot
increase the cost of the tree, but it gives a tree that has one more edge in
common with your tree.  This contradicts the choice of the least cost tree, so
there must have been no tree of lower total cost than the one you constructed.}
\ep

The method you used in Problem \ref{mincostspantree} is called a {\em
greedy method}\index{greedy method}, because each time you made a choice of
an edge, you chose the least costly edge available to you.

\subsection{The deletion/contraction recurrence for spanning trees} 

There are two operations on graphs that we can apply to get a
recurrence (though a more general kind than those we have studied for
sequences) which will let us compute the number of spanning trees of a
graph.  The operations each apply to an edge $e$ of a graph $G$.  
The first is called {\em deletion};\index{deletion} we {\em delete} the
edge
$e$ from the graph by removing it from the edge set.  Figure
\ref{twodeletions} shows how we can delete edges from a graph to get a
spanning tree.
\begin{figure}[htb]\caption{Deleting
two appropriate edges from this graph gives a spanning
tree.}\label{twodeletions}
\begin{center}\mbox{\psfig{figure=twodeletions.eps,height=.5in
}}
\end{center}\end{figure}

  The second operation is called {\em
contraction}.
\begin{figure}[hbt]\caption{The results of
contracting three different edges in a
graph.}\label{threecontractions}\vglue-1in
\begin{center}\mbox{\psfig{figure=threecontractions.eps,height=2in
}}
\end{center}
\end{figure} 
  Contractions of three
different edges in the same graph are shown in Figure
\ref{threecontractions}. Intuitively, we contract an edge by shrinking it
in length until its endpoints coincide; we let the rest of the graph ``go
along for the ride.''  To be more precise, we
{\em contract}\index{contraction} the edge
$e$ with endpoints
$v$ and
$w$ as follows:
\begin{enumerate}
\item remove all edges having either
$v$ or $w$ or both as an endpoint from the edge set, 
\item remove $v$ and $w$ from the vertex set,
\item add a new vertex
$E$ to the vertex set, 
\item add an edge from $E$ to each remaining
vertex that used to be an endpoint of an edge whose other endpoint was
$v$ or $w$, and add an edge from $E$ to $E$ for any edge other than
$e$ whose endpoints were in the set $\{v,w\}$.
\end{enumerate}
 We use $G-e$ (read as $G$
minus $e$) to stand for the result of deleting $e$ from $G$, and we use
$G/e$ (read as $G$ contract $e$) to stand for the result of contracting
$e$ from $G$.
\bp
\itemesi How do the number of spanning trees of $G$ not containing the edge
$e$ and the number of spanning trees of $G$ containing $e$ relate to the
number of spanning trees of $G-e$ and $G/e$?  Use $\#(G)$ to stand for
the number of spanning trees of $G$ (so that, for example, $\#(G/e)$
stands for the number of spanning trees of $G/e$).  Find an expression
for
$\#(G)$ in terms of $\#(G/e)$ and $\#(G-e)$.  This expression is called
the {\em deletion-contraction recurrence}\index{deletion-contraction
recurrence}\index{recurrence!deletion-contraction}.  Use it to compute the
number of spanning trees of the graph in Figure
\ref{spantreeexercise}.  
\begin{figure}[htb]\caption{A
graph.}\label{spantreeexercise}%\vglue-1in
\begin{center}\mbox{\psfig{figure=spantreeexercise.eps,height=1.0in
}}
\end{center}
\end{figure}
\solution{The number of spanning trees of $G$ not containing $e$ is the number
of spanning trees of $G-e$.  The number of spanning trees of $G$ containing the
edge $e$ is the number of spanning trees of $G/e$.  Therefore $\#(G) =\#(G-e)
+\#(G/e)$.  Applying the formula twice to $G$ gives 
\begin{eqnarray*}\#(G)
&=&\#(G-\{1,2\}-\{2,3\}) + \#((G-\{1,2\})/\{2,3\})\\ &+&
\#((G/\{1,2\})-\{2,3\}) +
\#(G/\{1,2\}/\{2,3\}).
\end{eqnarray*}
We now show the four graphs on the right hand side of the equation.
\begin{center}\mbox{\psfig{figure=spantreeexerciseresult.eps%,height=1.0in
}}
\end{center}
We could now convert each of these graphs to trees with multiple edges by
deleting and contracting one more edge, say edge $\{1,5\}$, which would make
the analysis easier but the picture twice as big.  Since we can easily count
spanning trees of a triangle, we can also stop here, noting that the first
graph has three spanning trees, the second has six, the third has five, and
the fourth has seven, so the total number of spanning trees is $3+6+5+7=21$.}
\ep 



\subsection{Shortest paths in graphs}  Suppose that a company has a
main office in one city and regional offices in other cities.  Most of
the communication in the company is between the main office and the
regional offices, so the company wants to find a spanning tree that
minimizes not the total cost of all the edges, but rather the cost of
communication between the main office and each of the regional offices.
It is not clear that such a spanning tree even exists.   This problem is
a special case of the following.  We have a connected graph with nonnegative
numbers assigned to its edges.  (In this situation these numbers are often
called weights.)  The {\em (weighted) length}\index{path!length
of}\index{length (of a path)} of a path in the graph is the sum of the weights
of its edges.  The {\em distance}\index{distance in a weighted
graph}\index{graph!distance in} between two vertices is  the least
(weighted) length of any path between the two vertices.  Given a vertex
$v$, we would like to know the distance between $v$ and each other
vertex, and we would like to know if there is a spanning tree in $G$ such
that the length of the path in the spanning tree from $v$ to each vertex
$x$  is the distance from $v$ to $x$ in $G$.
\bp \item\label{Dijkstra}
Show that the following algorithm (known as
Dijkstra's\index{Dijkstra's algorithm}\index{distance in a graph}
algorithm) applied to a weighted graph whose vertices are labelled 1 to
$n$ gives, for each
$i$, the distance from vertex 1 to i as
$d(i)$.
\begin{enumerate}
\item Let $d(1) = 0$. Let $d(i) = \infty$ for all other $i$.  Let
$v(1)$=1. Let
$v(j) = 0$ for all other
$j$.  For each $i$ and $j$, let $w(i,j)$ be the minimum weight of an edge
between $i$ and $j$, or $\infty$ if there are no such edges.  Let
$k=1$.  Let $t=1$.
\item For each $i$, if $d(i)>d(k) + w(k,i)$ let $d(i)= d(k) +w(k,i)$.
\item Among those $i$ with $v(i)=0$, choose one with $d(i)$ a minimum,
and let $k=i$.  Increase $t$ by 1. Let $v(i) =1.$
\item Repeat the previous two steps until $t=n$ 
\end{enumerate} 
\solution{We prove that the distance from vertex 1 of a vertex $i$ with
$v(i)=1$ is $d(i)$.  We use induction on the number $t$ of vertices with
$v(i)=1$. If $t=1$, then $d(1)=0$ is the distance from vertex 1 to vertex 1. 
Now when $t=s-1$, we have vertices $i_1$, $i_2$, \ldots $i_{s-1}$ such that
$v(i_p)=1$.  Suppose inductively that  $d(i_p)$ is the distance from
vertex
$i_p$ to vertex 1 for
$p=1,2,\ldots,s-1$. When $t=s$, we choose a vertex $i$ with $d(i)$ a minimum. 
Suppose $u_1,u_2,\ldots, u_r$ is the sequence of vertices of a shortest (least
total weight) path from vertex $u_1=1$ to vertex
$u_r=i$ and that the length (total weight) of this path is less than 
$d(i)$. Suppose that some vertex $u_p$ has $v(u_p)=0$, and choose the smallest
$p$ such that this is so.  Then $d(u_{p-1})$ is the distance from vertex $1$ to
vertex $u_{p-1}$, and $d(u_{p-1}) + w(u_{p-1},u_p)$ is less than $d(i)$
because the length of the path from $u_1$ to $u_r$ is less than $d(i)$. 
Then we would not have chosen the vertex $i$ after all, but rather the
vertex $u_p$, which is a contradiction, so all the vertices $u_p$ with $p<r$
must have
$v(u_p)=1$, and, by our inductive hypothesis, $d(u_p)$ must be the distance
from vertex 1 to vertex $u_p$ for $p<r$.  Thus when we were computing $d(i)$,
we would have found that $d(i)\le d(u_{r-1}) +w(u_{r-1},i)$.  Thus $d(i)$
must be the distance from vertex 1 to vertex $i$ after all.  Therefore the
fact that Dijkstra's algorithm works when $t=s-1$ implies that it works when
$t=s$, so that, by the principle of mathematical induction, it works for all
nonnegative integers
$t$.}

\item Is there a spanning tree such that the distance from vertex $1$ to
vertex $i$ given by the algorithm in Problem \ref{Dijkstra} is the
distance for vertex 1 to vertex $i$ in the tree (using the same weights
on the edges, of course)?
\solution{Yes, when we choose the $i$ with $d(i)$ a  minimum, before we change
$k$ to $i$, we add an edge from vertex $k$ to vertex $i$ to the edge set of a
graph on the vertex set $[n]$.  We get a tree each time we do this step,
because we are adding a vertex of degree 1 to a smaller tree.  We essentially
proved in the solution to Problem \ref{Dijkstra} that the path from vertex 1
to vertex
$i$ in this tree has length (total weight) $d(i)$.  Using the same approach we
could prove it directly by induction on the number of vertices in our tree.}
\ep\vfil\break\vfilneg

\section{Supplementary Problems}
\begin{enumerate}
\item Use the inductive definition of $a^n$ to prove that $(ab)^n=a^nb^n$ for
all nonnegative integers $n$.
\solution{If $n=0$ we get $(ab)^0=1$ and $a^0b^0=1$.  Assume inductively that
$(ab)^{n-1}=a^{n-1}b^{n-1}$.  Then by the inductive definition, inductive
hypothesis,  and commutative law, 
$$(ab)^n= (ab)^{n-1}ab=a^{n-1}b^{n-1}ab=a^{n-1}ab^{n-1}b=a^nb^n.$$
Thus the fact that $(ab)^{n-1}=a^{n-1}b^{n-1}$ implies the fact that $(ab)^n=
a^nb^n$.  Therefore by the principle of mathematical induction,
$(ab)^n=a^nb^n$ for all nonnegative integers $n$.}

\item Give an inductive definition of $\displaystyle \bigcup_{i=1}^nS_i$ and
use it and the two set distributive law to prove the distributive law
$\displaystyle{A\cap \bigcup_{i=1}^n S_i=\bigcup_{i=1}^n A\cap S_i}$.
\solution{We define $\displaystyle\bigcup_{i=1}^1S_i= S_i$ and $\displaystyle
\bigcup_{i=1}^n S_i =
\bigcup_{i=1}^{n-1}S_i
\cup S_n$.  Then $$\displaystyle A\cap\bigcup_{i=1}^1S_i=
A\cap S_1=\bigcup_{i=1}^1 A\cap S_i.$$ Assume that $n>1$ and
$\displaystyle A\cap\bigcup_{i=1}^{n-1}=\bigcup_{i=1}^{n-1}A\cap S_i$.  Now
\begin{eqnarray*}
A\cap\bigcup_{i=1}^nS_i &=& A\cap\left(\bigcup_{i=1}^{n-1}S_i
\cup S_n\right)=\left(A\cap\bigcup_{i=1}^{n-1}S_i\right) \cup\left( A\cap
S_n\right)\\&=&\left(\bigcup_{i=1}^{n-1}A\cap
S_i\right)\cup\left(A\cap S_n\right)=\bigcup_{i=1}^n A\cap S_i.
\end{eqnarray*}
  Thus the
truth of the distributive law for distributing an intersection over a union of
$n-1$ sets implies its truth for distributing an intersection over a union of
$n$ sets.  Therefore by the principle of mathematical induction, the
distributive law
$\displaystyle A\cap\bigcup_{i=1}^nS_i=
\bigcup_{i=1}^nA\cap S_i$ holds for all positive integers $n$.}

\itemi A hydrocarbon molecule is a molecule whose only atoms are
either carbon atoms or hydrogen atoms.  In a simple molecular model of a
hydrocarbon, a carbon atom will bond to exactly four other atoms and
hydrogen atom will bond to exactly one other atom.  We represent a
hydrocarbon compound with a graph whose vertices are labelled with C's
and H's so that each C vertex has degree four and each H vertex has
degree one.  A hydrocarbon is called an ``alkane" (common examples are
methane (natural gas), propane, hexane (ordinary gasoline), octane (to
make gasoline burn more slowly), etc.) if the graph is a tree.  How many
vertices are labelled
$H$ in the graph of an alkane with exactly
$n$ vertices labelled
$C$?
\solution{We have $n$ vertices of degree four, and so if we have $m$ vertices
of degree 1, we get $4n+m=2(m+n-1)$ from the fact that the sum of the degrees
of the vertices must be twice the number of edges. Thus we have $m=2n+2$
hydrogen atoms.}
\iffalse
How many different alkanes have exactly
$n$ vertices labelled
$C$?  (Here we say two trees are the same if we can make their drawings
congruent by shortening and lengthening lines, or moving  the vertices
and edges around, making sure that after we move things around, the edges
are attached to the same vertices as before.)\fi

\item \begin{enumerate}
\item Give a recurrence for the number of ways to divide
$2n$ people into sets of two for tennis games.  (Don't worry about who
serves first.) \solution{$t_{2n}=(2n-1)t_{2n-2}$}
 
\itemi Give a recurrence for the number of ways to divide $4n$
people into sets of four for games of bridge.  (Don't worry about how
they sit around the bridge table or who is the first dealer.)
\solution{$b_{4n}={4n-1\choose 3}b_{4n-4}$.}
\end{enumerate}

\item Use induction to prove your result in Supplementary Problem
\ref{composition:numberof} at the end of Chapter 1.
\solution{A composition of $n$ is an ordered list of positive numbers that
adds to $n$. We wish to show that there are $2^{n-1}$ compositions of $n$.
There is one composition of the number 1, and
$2^{1-1}=1$.  Now assume inductively that there are $2^{n-2}$ compositions of
the number
${n-1}$.  From a composition of $n-1$, we can get a composition of $n$ either
by making a new last part of size 1, or by adding one to the last part. 
Clearly these two operations give different partitions of $n$; what is not so
clear is that they give all partitions of $n$, but they do:  Either the last
part of a partition of $n$ is 1, in which case it comes from the first kind of
operation, or it is larger than one, in which case it comes from the second
operation.  Thus the number of compositions of $n$ is twice the number of
compositions of $n-1$, and so is $2\cdot2^{n-2}=2^{n-1}$.  Therefore the
statement that there are $2^{n-2} $ compositions of $n-1$ implies the statement
that there are $2^{n-1}$ compositions of $n$.  Thus by the principle of
mathematical induction, there are $2^{n-1}$ compositions of $n$ for every
positive integer $n$.}

\item Give an inductive definition of the product notation $\displaystyle
\prod_{i=1}^n a_i$.  \label{inductiveprodnotation}
\solution{$\displaystyle\prod_{i=1}^1a_i=a_1$, and $\displaystyle\prod_{i=1}^n
a_i= \left(\prod_{i=1}^{n-1}a_i\right)\cdot a_n$.}

\item Using the fact that $(ab)^k =a^kb^k$, use your inductive definition
of product notation in Problem \ref{inductiveprodnotation} to prove that
$\displaystyle \left(\prod_{i=1}^n a_i\right)^k=\prod_{i=1}^n a_i^k$.
\solution{When $n=1$ we get $\displaystyle\left(\prod_{i=1}^1 a_i\right)^k=
a_i^k =\prod_{i=1}^1a_i^k$.  Now assume inductively that
$\displaystyle\left(\prod_{i=1}^{m-1}a_i\right)^k=\prod_{i=1}^{m-1}a_i^k$. 
Then we may write
$$\left(\prod_{i=1}^ma_i\right)^k=\left(\left(\prod_{i=1}^{m-1}
a_i\right)\cdot a_m\right)^k=\left(\prod_{i=1}^{m-1}
a_i^k\right)\cdot a_m^k=\prod_{i=1}^m a_i^k.$$  Thus the correctness of  the
formula for $n=m-1$ implies its correctness for $n=m$.  Therefore by the
principle of mathematical induction, the formula holds for all positive
integers $n$.}

\itemi How many labelled trees on $n$ vertices have exactly four vertices of
degree 1?
\solution{The vertices of degree 1 are the vertices that do not appear in the
Grey code for the tree.  So we first choose four vertices out of $n$ in
$n\choose 4$ ways to be our vertices of degree 1, and then we use the
remaining $n-4$ vertices to fill in our list of $n-2$ vertices, using each of
the $n-4$ at least once.  Thus we either use one of them 3 times and the rest
once, or two of them twice and the rest once. There are $n-4$ ways to choose
the one we use three times and ${n-2\choose 3}{n-5\choose
1}{n-6\choose1}\cdots{1\choose1}={(n-2)!\over3!}$ ways to label the
$n-2$ places with the chosen vertices.  There are $n-4\choose 2$ ways to choose
the vertices we would use twice, and ${n-2\choose
2}{n-4\choose2}{n-6\choose1}{n-7\choose1}\cdots {1\choose1}={(n-2)!\over 2!2!}$
ways to assign the chosen vertices to the
$n-2$ places in the Pr\"ufer Code.  Thus we have 
\begin{eqnarray*}
&&{n\choose 4}\left((n-4){(n-2)!\over 3!} +{(n-4)(n-3)\over
2}{(n-2)!\over 4}\right)\\
&=&{n!\over 48}(n-2)^{\underline{3}}\left({1\over3} +{n-3\over 4}\right)
\end{eqnarray*} 
possible Pr\"ufer codes and therefore the same number of
labelled trees.}

\itemih The degree sequence of a tree is a list of the degrees of the
vertices in nonincreasing order.  For example the degree sequence of the
first graph in Figure \ref{spanningtrees} is 43221.  How many labelled
trees are there on $n$ vertices with degree sequence $d_1,d_2,\ldots d_n$, or,
equivalently, with the degree sequence in which the degree $d$ appears $i_d$
times?
\solution{
We first solve the simpler problem in which we assume $d_i$ is the
degree of vertex $i$.  The number of times
$i$ appears in the Pr\"ufer code of a tree is one less than the degree of $i$,
so vertex
$i$ appears
$d_i-1$ times.  Thus the sum of the $d_i-1$ should be $2n-2-n=n-2$.  Of the
$n-2$ places in the Pr\"ufer code, we want to label $d_1-1$ of them with 1,
$d_2-1$ of them with 2 and in general $d_i-1$ of them with $i$.  There are
$${n-2\choose
d_1-1}{n-2-(d_1-1)\choose d_2-1}{n-2-(d_1-1+d_2-1)\choose
d_3-1}\cdots{d_n-1\choose d_n-1}$$ ways to do this, so the number of trees
in which vertex $i$ has degree $d_i$ is ${(n-2)!\over
(d_1-1)!(d_2-1)!\cdots(d_n-1)!}$.

Now we modify the solution by observing that to count all graphs with a given
degree sequence, the actual vertices which have the given degrees is
irrelevant, so we must multiply the result of the easier problem by the number
of ways to assign the degrees to the vertices.  To assign the degrees, we can
list the vertices in $n!$ ways, choose the first $i_1$ of these vertices to
have degree 1, then next $i_2$ to have degree 2, and so on.  But the order
in which we list the vertices of a given degree is irrelevant.  Thus the number
of ways to assign the degrees is
$n!\over i_1!i_2!\cdots i_n!$.  Once the degrees are assigned, there are
$(n-2)!\over
\prod_{d=1}^n (d-1)!^{i_d}$, by translating our easier result.  Thus the total
number of trees with the degree sequence in which there are $i_d$ vertices of
degree $d$ is
$$n!(n-2)!\over\prod_{j=1}^n i_j!(j-1)!^{i_j}.$$}

\end{enumerate}